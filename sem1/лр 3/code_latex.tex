\documentclass[10 pt]{extarticle}
\usepackage{multicol}
\usepackage[a4paper, total={6.6in, 10in}]{geometry}
\setlength{\columnsep}{0.5cm}
\usepackage{parskip}
\setlength{\parskip}{0ex}
\setlength{\parindent}{14pt}
\setcounter{page}{135}

\begin{document}
\begin{multicols}{2}
\setlength{\parindent}{24pt}
\setlength{\parskip}{0pt}
for ensuring the consistency and interoperabil- \par
ity of this information and different types of ac- \par
tivities that use this information, it is proposed \par
to use an ontological approach based on the rep- \par
resentation of information in the form of seman- \par
tic networks (knowledge graphs);
\setlength{\parindent}{14pt}
\begin{itemize}
    \item  An agent-based approach is proposed as tools for
retrieving already existing information, reusing it for
problem solving and accumulating it from various
information sources;
    \item Semantic user interfaces using ontological, grapho-dynamic and agent-based approaches are proposed
as means to realise personalised representation and
transfer of information.
\end{itemize}
\setlength{\parskip}{3 pt}

    Such an information resource can be realised in the
form of an \textbf{\textit{Electronic glossary}} for all available spheres
of human activity in society. In the traditional sense,
such a glossary can be a dictionary of highly specialised
concepts and their terms in various subject domains, the
text of which will be clear to the end user. In a broader
sense, such a glossary will be understood as a system or
subsystem with the help of which:

\begin{itemize}
    \item you can \underline{quickly find} quickly find already existing information:
concepts, their terms, definitions, connections with
other concepts, and so on;
    \item \setlength{\parskip}{0pt} \underline{reuse} reuse information by both humans and computer
systems that know how to communicate with this
glossary;
    \item \underline{accumulate} information, i.e. new information from
various sources can be entered both manually and
automatically;
    \item information can be visualised depending on the
user’s learning level.
\end{itemize}

\setlength{\parskip}{3 pt}

    The \textbf{\textit{OSTIS Technology}} is proposed to be used as a
technology that allows to realise such systems and has all the necessary methods and means for their implementation. The \textit{OSTIS Technology} is an open semantic technology of complex life cycle support of semantically interoperable intelligent computer systems of new generation. The purposefulness of using this technology is determined by the tasks that can be solved with the help of this technology, and the high level of scientific problem, which is aimed at solving this technology.

\setlength{\parskip}{0 pt}

The \textit{OSTIS Technology} is known for its basic principles [34], [35], [36]:
    

\begin{itemize}
    \item \setlength{\parskip}{3 pt} all knowledge is described by means of a unified knowledge
representation language — \textit{SC-code}, which ensures syntactic and semantic interoperability of this knowledge and makes it possible to interpret knowledge not only by humans but also by computer systems;
    \item \setlength{\parskip}{0 pt} All knowledge is structured by means of a hierarchy of subject domains 
and their corresponding ontologies, through which the consistency of this knowledge is ensured;
    \item knowledge processing of various kinds is based on the principles of
graphodynamic models, with the help of which it is possible to understand the
meaning of this knowledge efficiently and flexibly;
    \item  all knowledge is accumulated in the form of semantically powerful
libraries of reusable components, with the help of which the reuse of already existing knowledge in problem solving is realised;
    \item  all knowledge is open and transparent in use and
modification by both humans and systems.
\end{itemize}

\setlength{\parskip}{3 pt}

\textit{SC-code} is the main internal formal universal abstract language for
representing information constructs in ostis-systems. \textit{SC-code} supports various data types, including numbers, strings, lists and other data structures. It also provides facilities for working with knowledge bases and performing logical operations. There are 3 main external languages of ostis-systems: SCs-code, SCn-code and SCg-code [37]. They provide a way for ostis-systems to communicate with their users and other ostis-systems.

\setlength{\parskip}{0 pt}

This technology is considered to be a new generation
technology. And in the context of the \textit{OSTIS Technology},
all computer systems, in particular intelligent systems,
developed on the basis of this technology are called
new generation intelligent systems or ostis-systems. The
\textit{OSTIS Technology} is realised in the form of a special
ostis-system, which is called the \textit{OSTIS Metasystem} [38]
and the knowledge base, which this system contains:

\setlength{\parskip}{3 pt}
\begin{itemize}
    \item The formal theory of ostis-systems;
    \item \setlength{\parskip}{0 pt} The standard of ostis-systems (standard of ostis-systems knowledge
bases, ostis-systems problem solvers, ostis-systems interfaces);
    \item The standard of methods and tools for ostis-systems
life cycle support (the core of the Library of
reusable ostis-system components (the OSTIS Library), methods for supporting the life cycle of ostis-systems and their components, tools for supporting
the life cycle of ostis-systems).
\end{itemize}

\setlength{\parskip}{3 pt}

    \textit{The OSTIS Standard} defines general principles and rules for the
development and use of knowledge bases and intelligent systems based on the \textit{OSTIS Technology}.It defines the structure of knowledge bases, data formats, ways of knowledge organisation and other aspects that should be taken into account when developing intelligent systems based on the \textit{OSTIS Technology}. In addition, the \textit{OSTIS Standard} defines the rules of organisation and representation of knowledge in knowledge bases. The \textit{OSTIS Standard} helps to ensure interoperability and uniformity within the projects using the \textit{OSTIS Technology} and provides convenience for developers and researchers to work together.

\setlength{\parskip}{0 pt}

In the context of this paper, the \textit{OSTIS Metasystem} [38], [39] attracts attention because it is some computer version of the \textit{OSTIS Standard}, additionally implementing tools for processing, visualising and using this
standard, the aim of which is to provide a consistent and compatible activity both for the development of
intelligent systems and for the improvement of methods,
models and tools for their development, which is a subobjective of the Electronic glossary conceived by this
work.

In addition, all ostis-systems are combined to form
an integrated ecosystem — the \textit{OSTIS Ecosystem} [40],
containing knowledge from all subject domains of society
in a consistent, semantically interoperable and understandable form. If we compare the Electronic glossary
and the \textit{OSTIS Ecosystem}, it is obvious that the Electronic
glossary is a variant of the \textit{OSTIS Ecosystem} knowledge
base. However, in the context of this work, it is more
appropriate to specify the Electronic glossary to the
glossary, which is a variant of displaying the \textit{OSTIS Standard}, because if we consider the Electronic glossary as
a variant of displaying the knowledge base of the \textit{OSTIS
Ecosystem}, it is necessary in this work to focus in more
detail not on the principles of structuring information in
this glossary, but on the principles of coordination of
information from different subject domains, which are
united in the form of a single knowledge base of the
\textit{OSTIS Ecosystem} and which is the subject of the \textit{OSTIS
Standard}.

Also due to the identified dependence of the Electronic glossary on the \textit{OSTIS Standard} and the \textit{OSTIS Technology} as a whole, we will refer to this Electronic
glossary as the \textbf{\textit{OSTIS Glossary}}. And also it is important
to understand that in this case the \textit{OSTIS Glossary} is not
some separate ostis-system, but is a form of display of the
\textit{OSTIS Standard}, its part, visualised according to certain
formalised rules, and by virtue of this it is also a part of
the knowledge base of the \textit{OSTIS Metasystem}, to which
it is possible to ask various questions.

Therefore, in the following sections, besides the content of the conceived the \textit{OSTIS Glossary}, the rules of
its development, visualisation and tools for working with
it will be described in detail. That is, in this paper it is
important to fulfil the following tasks:

\setlength{\parskip}{3 pt}

\begin{itemize}
    \item describe and fix the principles and rules of development and consistency of the \textit{OSTIS Glossary}
fragments both with itself and with the \textit{OSTIS Standard};
    \item \setlength{\parskip}{0 pt} describe and fix the rules of placement, structuring,
identification of concepts and fragments of concepts
in the \textit{OSTIS Glossary};
    \item  describe and fix the principles of interaction of
users and systems with the \textit{OSTIS Glossary} and the
rules of visualisation of concepts and fragments of
concepts in it;
    \item he current structure of the \textit{OSTIS Glossary}.
\end{itemize}

\setlength{\parskip}{3 pt}

Further on we will consider in detail the principles of
consistency of the \textit{OSTIS Glossary} and the \textit{OSTIS Standard}, the rules of structuring and specification of \textit{OSTIS Glossary} objects, as well as the rules of identification of these objects within the \textit{OSTIS Glossary}.  \break

\setlength{\parskip}{0 pt}

\noindent{\textit{B. Principles of consistency of the OSTIS Glossary and
the OSTIS Standard}} \break

As stated earlier, the \textit{OSTIS Glossary} is closely related
to the \textit{OSTIS Standard}. This is due to the following
reasons, which at the same time are the \textbf{basic principles
for the development of the OSTIS Glossary}, consistency
of its text with itself and the text of the \textit{OSTIS Standard}:

\setlength{\parskip}{3 pt}

\begin{itemize}
    \item The \textit{OSTIS Glossary} is not a separate knowledge
base or ostis-system. On the contrary, the \textit{OSTIS
Glossary} is a semantically compatible and ordered
fragment of the \textit{OSTIS Standard}, some variant of
its display, which describes with sufficient detail
the entities and concepts used in Artificial Intelligence and, in particular, \textit{OSTIS Technologies}, as
well as the relations between them, references to
bibliographic sources and authors of these entities
and concepts. The \textit{OSTIS Glossary} is the result of
a collective consistency of terms both within the
\textit{OSTIS Technology} and across Artificial Intelligence.
    \item \setlength{\parskip}{0 pt} In addition to the \textit{OSTIS Glossary}, an important
component of the \textit{OSTIS Standard} is the \textit{OSTIS
Bibliography}. The \textit{OSTIS Bibliography} is a list of
the literature used in the \textit{OSTIS Standard}. The
\textit{OSTIS Bibliography} is the result of the analysis
of other works and analogues studied during the
development of the \textit{OSTIS Technology} and includes
brief bibliographic descriptions of both other technologies similar to the \textit{OSTIS Technology} and the
technologies, models and tools on which the \textit{OSTIS
Technology} itself is based.
    \item  Consequently, the key objects of description in the
Glossary may be:
        \begin{itemize}
            \item  concepts (absolute and relative);
            \item specific entities that are not concepts, e.g. specific
systems, projects, technologies, languages, etc.
(the \textit{OSTIS Ecosystem} [40], the \textit{OSTIS Metasystem}, Neo4j Project, RDF Language);
        \item  specific individuals;
        \item  specific bibliographic sources (books, articles,
electronic resources).
        \end{itemize}
    \item  The \textit{OSTIS Glossary} is not a static structure stored
in the knowledge base of the \textit{OSTIS Metasystem}. It is the result of the work of some collective of agents transforming the hierarchy of sections of the \textit{OSTIS Standard} into some simplified from the reader’s point of view hierarchy of sections, in which concepts are ordered lexicographically rather than logically, i.e. the \textit{OSTIS Glossary} can be formed by means of explicit or indirect start of a non-atomic agent consisting of agents: forming a particular section of this glossary, concatenation of glossary sections, filtering of concepts and their
specifications according to given criteria, and so on.
    \item  Since the \textit{OSTIS Glossary} is some fragment of the
knowledge base of the \textit{OSTIS Metasystem}, it is more
appropriate to develop it by the same means that
are used to develop any knowledge base of the ostissystem. From this point of view, the development of
the \textit{OSTIS Glossary} is reduced to the development
of the knowledge base of the \textit{OSTIS Metasystem},
including the \textit{OSTIS Standard}, already loaded in
this knowledge base. Thus, manual transformation
of the \textit{OSTIS Standard} into the \textit{OSTIS Glossary}
is not required and is automated by existing ostissystem knowledge base development tools.
    \item  For visualisation of the \textit{OSTIS Glossary} the existing
tools for development of ostis-systems knowledge
bases are sufficient. Viewing of the \textit{OSTIS Glossary}
from the whole knowledge base of the \textit{OSTIS Metasystem} should be done with the help of a specialised
agent. Such an agent should allow to display it in
one of the external sc-languages (SCn-code, or SCg-code).
\end{itemize}

\setlength{\parskip}{3 pt}

In other words, the OSTIS Glossary should not be
some other text describing the current state of the OSTIS
Technology, on the contrary, the Glossary should be
consistent and semantically interoperable with the OSTIS
Standard and should be only some form of its presentation, simplified for the reader, reducing the threshold for
new people to enter the OSTIS Technology, allowing to
quickly find concepts and agree new ones. \break

\setlength{\parskip}{0 pt}

\noindent{\textit{C. Rules for structuring and specification of the OSTIS Glossary objects}} \break

The main purpose of the \textit{OSTIS Glossary} is to present
concise specifications of the \textit{OSTIS Standard} concepts
in an organised form, simplified for the reader. The text
of the \textit{OSTIS Standard} is presented in the form of a
sequence of ordered and organised sections of subject
domains and ontologies containing a logical statement
of the specifications of the objects considered within the
\textit{OSTIS Technology}. The \textit{OSTIS Standard} is based on the
\textbf{\textit{following principles of text structuring}}:

\setlength{\parskip}{3 pt}

\begin{itemize}
    \item The \textit{OSTIS Standard} is the main part of the knowledge base of the \textit{OSTIS Metasystem} and is a description of the current state of the \textit{OSTIS Technology}.
    \item \setlength{\parskip}{0pt}As a formal language for the external representation
of the \textit{OSTIS Standard}, SCn-code, which is an
external form of \textit{SC-code} representation, is used.
    \item The \textit{OSTIS Standard} is ontologically structured, i.e.
it is a hierarchical system of related formal subject
domains and their corresponding formal ontologies,
thus ensuring a high level of stratification of the
\textit{OSTIS Standard}.
    \item  Each subject domain can be matched:
        \begin{itemize}
            \item a family of corresponding ontologies of different
kinds;
            \item  a set of semantic neighbourhoods describing the
research objects of this subject domain.
        \end{itemize}
subject domains are the basis for structuring the
sense space, a means of localisation, focusing attention on the properties of the most important
classes of described entities, which become classes
of objects of research in subject domains.
    \item Each concept used in the \textit{OSTIS Standard} has its
own place within this standard, its own subject
domain and its corresponding ontology, where this
concept is investigated in detail, where all the basic information about this concept and its various
properties is concentrated.
    \item The \textit{OSTIS Standard} also includes files of information constructs that are not \textit{SC-code} constructs
(including sc-texts belonging to different natural
languages). Such files allow to formally describe
in the knowledge base the syntax and semantics
of various external languages, and also allow to
include in the knowledge base various explanations,
notes addressed directly to users and helping them
to understand the formal text of the knowledge base.
    \item From a semantic point of view, the \textit{OSTIS Standard}
is a large refined semantic network, which is nonlinear in nature and which includes signs of all kinds
of described entities (material entities, abstract entities, concepts, relations, structures) and, accordingly,
contains links between all these kinds of entities
(in particular, links between links, links between
structures).
    \item The \textit{OSTIS Standard} is a hierarchical system of
subject domains and their corresponding ontologies
specifying these subject domains. Each of the subject domains describes the corresponding classes of
research objects with the maximum possible degree
of detail defined by a set of relations and parameters
defined on the classes of research objects.
    \item Each section of the \textit{OSTIS Standard} contains the
knowledge that is part of the subject domain and
ontology that is either fully represented by the
specified section or partially represented by the
specification of one or more specific objects of
study.
    \item The specification of each subject domain and each
section should have a sufficient degree of completeness. At a minimum, the role of each concept used
in each subject domain should be specified.
\end{itemize}

Since the \textit{OSTIS Glossary} is nothing but a part of the
|textit{OSTIS Standard}, the structuring principles of the \textit{OSTIS
Glossary} are the same as the structuring principles of
the \textit{OSTIS Standard}. The exception is that the structuring
of the \textit{OSTIS Glossary} should happen automatically, by
some agent generating a structure corresponding to this.

\end{multicols}
\end{document}

\maketitle

\section{}

\end{document}

