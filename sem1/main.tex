\documentclass{scndocument}
\pagestyle{plain}
\usepackage{import}
\usepackage{scn}
\usepackage{amsmath} 
\usepackage{multicol}
\usepackage{ragged2e} %выравнивание текста по ширине
\backgroundsetup{contents={}}
\setcounter{page}{66}
\begin{document}
\begin{SCn}
\begin{multicols}{2}
\begin{itemize}
\item $n_{s_l{}_v} \in (\bar{\mathbb{N}} \cup \{0\})$  — the index of the last vacant segment in sc-memory;
\item $n_{s_l{}_r} \in (\bar{\mathbb{N}} \cup \{ 0 \})$  — the index of the last released segment in sc-memory;
\item $m_s \in M$  — the object that synchronizes access to $S, n_{s_l{}_e}$  and $n_s{}_{max}$;
\item  $m^n {_{s_l{}_v}} \in M$  — the object that synchronizes access to  $n_{s_l{}_v}$;
\item  $m^n{_{s_l{}_r}} \in M$ — the object that synchronizes access to  $n_{s_l{}_r}$;
\item \textit{SSPI = \{engage, release \}} — interface of sc-elements storage in sc-memory.
\end{itemize}
\justifying All allocated segments \textit{S} may be free \textit{$S_e$} or engaged \textit{$S_v \subseteq S$} . The set of free sc-memory segments \textit{${S_f}$} includes the set of vacant segments \textit{${S_v}\subseteq {S_f}$}  and the set of released segments \textit{${S_r} \subseteq {S_f}$}.
\justifying 
Cells in sc-memory segments  \textit{E} may be free {${E_f}\subseteq {E}$} and engaged \textit{${E_e}\subseteq {E}$}.  The set of free sc-memory cells $E_f = \{{e^{ij}{_{f}}} | {e^{ij}{_{f}}  \epsilon s^i{_f} ,  s^i{_f}  \epsilon  S_f \}}$ includes the setof vacant cells $E_v \subseteq E_f$ and the set of released cells $ E_r \subseteq E_f.$


Consequently, the following statements hold for all sc-memory segments and cells in them:
\begin{itemize} 


\item \textit {$E_f \cup  E_e = E, E_f \cap  E_e  = \oslash$},

\item \textit {$E_f \cup  E_e = E, E_f \cap  E_e = \oslash$},

\item \textit {$S_f \cup  S_e = E, S \cup  E_e = \oslash$},

\item \textit {$S_v \cup  S_r = S_f, S_v \cup S_r \subseteq S_f $},
\end{itemize}
For the sets of engaged and released cells, the corre-sponding transitions can be defined in the form of:
\begin{itemize} 
\item operation of allocating a \textit{engage}: {$E_f \xrightarrow{} E_e$}, which
changes the state of the cell from "released" to
"engaged":
\end{itemize}
\begin{equation*} 
engage() = 
 \begin{cases}
 e_{ij} \in E_e, \text{if} \ \exists e_{ij} \in s_i, E_f \wedge n \leq n_s{}_{max} \\
\text{Error}, \qquad \text{if} \    n > n_s{}_{max} \vee |E_f| = 0;
 \end{cases}
\end{equation*}
\begin{itemize} 
\item operation of releasing a cell in sc-memory segment
\textit{release} : {$E_e \xrightarrow{} E_f$} , which changes the state of the
cell from "engaged" to "released":
\end{itemize}
\begin{equation*}
release(e_{ij}) = 
 \begin{cases}
 e_{ij} \in E_f, \text{if} \ e_{ij} \in E_e, \\
\text{Error}, \qquad \text{if} \  e_{ij} \in E_e.

 \end{cases}
\end{equation*}
The algorithm of the cell engaging operation in sc-memory segment (engage) can be described as follows:
\begin{itemize} 
\item  Step 1: Try to find any vacant segment $s_i$
in the set $S_v$: 

\item[-] If such a segment exists, go to step 2. \leftskip=1em
\item[-] If no such segment exists, skip to step 3.

\item Step 2: Engage a new cell in the found vacant\leftskip=0em
segment $s_i$:
\item[-]  Increase $n_{e_{le}}$ in segment $s_i$ by 1.\leftskip=1em
\item[-] Occupy the cell $e_{ij}$ with index $n_{e_{le}}$ in segment $s_i$.
\item[-] Return the address of the engaged cell$e_{ij}$ and
terminate.
\item  Step 3: Attempt to get a new segment from set \textit{S}:\leftskip=0em
\item[-] If the number of engaged segments $n_{s_{le}}$ is less than the maximum $ n_s{}_{max}$
, create a new segment $s_i$ (set $n_{e_{le}}$ as 0) and add it to \textit{S}.\leftskip=1em
\item[*] Increase nele in segment $s_i$ by 1. \leftskip=3em
\item[*] Occupy the cell $e_{ij}$ with index $n_{e_{le}}$ in segment $s_i$.
\item[*] Return the address of the engaged cell $e_{ij}$ and terminate.
\item[-] If the maximum number of segments $n_s{}_{max}$ is
reached, go to step 4.\leftskip=1em
\item Step 4: Try to get the released segment $s_i$ from the
set $S_r:$  \leftskip=0em
\item[-] If there is no such segment, report an error and
terminate.\leftskip=1em
\item[-]If such a segment exists, proceed to step 5.
\item Step 5: Engage a new cell in the found released
segment $s_i$:\leftskip=0em
\item[-] Get the last released cell $e_{ij}$ by its number in
segment $n_{e_r}$.\leftskip=1em
\item[-] Occupy the cell  $eij$ with index $n_{e_r}$ in segment  $s_i$.
\item[-] Update $n_{e_r}$ for the next released cell.
\item[-] Return the address of the engaged cell $e_{ij}$ and terminate.
\end{itemize}
The algorithm of the cell releasing operation in sc-memory segment (release) can be described as follows:
\begin{itemize} 
\item Step 1: Verify the correctness of the given address
of cell $e_{ij}$ : 
\item[-] If the cell address does not exist in sc-memory,
terminate with an error.\leftskip=1em
\item[-] If the cell address exists in sc-memory, proceed
to step 2.
\item  Step 2: Using the cell address, determine the corresponding segment $e_i$ of the cell in sc-memory and proceed to step 3.\leftskip=0em
\item Step 3: Release the cell $e_{ij}$ :
\item[-] Update the information about the cell $e_{ij}$  , mark
it as released.\leftskip=1em
\item[-] Update the number of the last released cell in
segment $n_{e_r}$.
\item[-] Go to step 4.
\item Step 4: Update the information about the released
segments.\leftskip=0em
\item[-] If the released cell was the first released cell in
the segment, update the information of the last released segment in sc-memory $n_{slr}$ .\leftskip=1em
\item[-] Go to step 5.
\item Step 5: Terminate.\leftskip=0em
\end{itemize}

The described algorithms may include synchronization mechanisms to ensure data integrity during multithreaded sc-memory accesses. All synchronization operations in these algorithms can be performed using appropriate synchronization objects $m_s ,  m_e,  m^n {_{s_l{}_v}} m^n {_{s_l{}_r}}$,  The model of the synchronization object will be discussed later. \par
 Basically, the advantages of the described algorithms
are due to the advantages of the cell engaging algorithm
in sc-memory, which are as follows:

\begin{itemize}
\item The cell engaging algorithm in sc-memory tries to
find a vacant segment before creating a new one. If
there is no vacant segment, it tries to create a new
one if the maximum number of segments has not
been reached. This approach avoids wasting memory
on creating unnecessary segments.
\item By updating the number of the last released cell in
a segment, the algorithm keeps track of which cells
are available for reuse. This tracking ensures that
the engaging process is fast and does not require
searching the entire memory for a released cell.
\end{itemize}
\par To analyze the complexity of these algorithms, let us
consider their main characteristics:
\begin{itemize}
\item In the algorithm of cell engaging operation in sc-memory: \item[-]checking the correctness of the cell address and
determining the appropriate segment can be accomplished in a time proportional to the number
\item[-]releasing a cell and updating segment information
requires a fixed number of operations.
of segments, or faster if an efficient data structure
is used to store the segments; \par
\end{itemize}
\par The complexity of the algorithms depends on the
data structures used and how they are processed. Assuming that multiple segments are processed efficiently,
for example using internal lists in released segments
and released segment cells, the underlying complexity
of the algorithms will be determined by the number of
operations required to process each step. Thus:
\begin{itemize}
\item The algorithm of cell engaging operation in sc-memory can have a complexity from \textit{O(1)}   to \textit{O(n)},
where \textit{n} is the number of segments, depending on
whether a free segment is found or a new segment
needs to be engaged;
\item The algorithm of cell releasing operation in scmemory basically has a complexity of \textit{O(1)}, since
most operations are performed in a fixed amount of
time, except for address correctness checking, which
may require \textit{O(n)} in the worst case.
\end{itemize}
\par So, an sc-element storage is a set of cells, each of
which can store some sc-element (be engaged) or can be
empty (free):
\[({\forall e \in E : (e \in E_e )\vee (e \in E_f)}).\]
\par Each cell $e_{ij} \in \textit{E} , e_{ij} \in  s_i , s_i \in$ \textit{S} has a unique internal address $\textit{a} = \langle i, j  \rangle \in A$. That is, the following statement always holds:
\[(\forall e_{ij} \in E, \exists ! a \in A : (e_i{}_j \in s_i) \wedge (s_i \in S) \wedge(a = \langle i, j  \rangle)).\]
\par Each cell stores either an sc-node \textit{N} or an sc-connector
\textit{C}:
\[({\forall e \in E : (e \in N )\vee (e \in EC)}).\]
\[N \cup C = E, N \cap C = \oslash\]
\par It is assumed that if a cell stores some sc-element, it
stores information characterizing this sc-element. Each
cell in sc-memory $e \in E$ can be represented as a tuple:
\[e =  \langle FI, EI, CI  \rangle\]
where
\begin{itemize}
\item \textit{FI} — characteristics of the stored sc-element, including the type of sc-element and its st
\item EI — information about sc-elements incident with
the given sc-element,
\item CI — information about the number of incoming
and outgoing sc-connectors for a given sc-element.
\end{itemize}
\par The characteristics of an sc-element \textit{FI} can be represented as:
\[FI =  \langle t,s  \rangle\]
where
\begin{itemize}
\item $t \in T$ is the syntactic type of a given sc-element
(e.g., sc-node, sc-connector, ostis-system file, etc.),
\item $s \in ES$ is the state of the cell (e.g., "engaged",
"free", etc.).
\end{itemize}
\[T =  T_n \cup T_c\]
$T_n = \{node, f ile\}$,$T_c = \{connector, arc, edge\}$,
where
\begin{itemize}
\item \textit{T} is the set of all possible syntactic types of sc-elements;
\item \textit {$T_n$} is set of all possible syntactic types of sc-nodes,
\textit{node, file} — actual sc-node label and ostis-system file label, respectively;
\item \textit {$T_c$} is the set of all possible syntactic types of
sc-connectors, \textit{connector, arc, edge} — the actual
sc-connector label, sc-arc label, and sc-edge label,
respectively;
\end{itemize}
\[ES = \{engaged, free\}\]
where
\begin{itemize}
    \item $ES$ is the set of all possible cell states;
\item $engaged$ — the cell is "engaged";
\item $free$ — the cell is "free".
\end{itemize}
\par Information about incident sc-connectors EI can be
represented as:
\[(\forall n \in N : ( n \ni EI = \langle b_o, b_i \rangle))\]
\[(\forall c \in C : ( c \ni EI =\]
= $\langle b_o, b_i, b, e, n_{bo}, p_{bo}, n_{bi}, p_{bi}, n_{eo}, p_{eo}, n_{ei}, p_{ei} \rangle))$
\\
where
\begin{itemize}
    \item  $b_o \in A$ is the sc-address of the initial sc-connector
outgoing from the given sc element,
\item $e_i \in A$ is the sc-address of the initial sc-connector incoming into the given sc-element,
\item $b \in A$ is the sc-address of the initial sc-element of
the sc-connector,
\item $e \in A$ is the sc-address of the final sc-element of
the sc-connector,
\item$ n_{bo} \in A$ is the sc-address of the next sc-connector
outgoing from the initial sc-element,
\item $p_{bo} \in A$ is the sc-address of the previous sc-connector outgoing from the initial sc-element,
\item $n_{bi} \in A$ is the sc-address of the next sc-connector incoming into the initial sc-element,
\item $p_{bi} \in A$ is the sc-address of the previous sc-connector incoming into the initial sc-element,
\item $n_{eo} \in A$ is the sc-address of the next sc-connector outgoing from the final sc-element,
\item $p_{eo} \in A$ is the sc-address of the previous sc-connector outgoing from the final sc-element,
\item $n_{ei} \in A$ is the address of the next sc-connector
incoming into the final sc-element,
\item $p_{ei} \in A$ is the address of the previous sc-connector
incoming into the final sc-element.
\end{itemize}
In this case, the following statements are true:\par
 $(\forall e \in E, \exists ! b_o,b_i \in A : (Inc(e, b_o) \wedge Inc(e,b_i))),$ \par
\[(\forall e \in C, \exists ! n_{bo}, p_{bo}, n_{bi}, p_{bi} \in A :\] 
\[((Inc(e,n_{bo}))\wedge(Inc(e,p_{bo}))\wedge (Inc(e,n_{bi}) \wedge Inc(e,p_{bi}))))\] \par
\[(\forall e \in C, \exists ! n_{o}, 
p_{eo}, n_{ei}, p_{ei} \in A :\]
$((Inc(e,n_{eo}))\wedge(Inc(e,p_{eo}))\wedge (Inc(e,n_{ei}) \wedge Inc(e,p_{ei}))))$
\par where $I_nc$ is the binary relation of incidence of two
sc-elements.
\par Information about the number of incoming and outgoing sc-connectors $C$ can be represented as
\[CI = \langle c_{in}, c_{out}\rangle\]
where
\begin{itemize}
    \item $c_{in} \in (\bar{\mathbb{N}} \cup {0})$ is the number of incoming sc-connectors in a given sc-element,
    \item $c_{out} \in (\bar{\mathbb{N}} \cup {0})$ is the number of outgoing sc-connectors from a given sc-element.
\end{itemize}
\par This information can be used to optimize the isomorphic search for sc-constructions over a given graphtemplate [2].
\par The model of storage of sc-elements in sc-memory
provides:
\begin{itemize}
    \item torage of sc-constructions, their sc-elements, characteristics and incid
    \item ability to create, modify, search and delete sc-elements
\end{itemize}
The advantages of this model are as follows:
\begin{itemize}
\item it provides efficient fragmentation and defragmentation of cells;
\item algorithms for allocating and freeing a memory
segment have asymptotic complexity from $O(1)$ to
$O(n)$, where n is the number of segments that must
be traversed to find a free segment.
\end{itemize}
\textit{B. Model of storage of external information constructions in sc-memory} \par
\textit{The model of storage of external information constructions in sc-memory} can be represented as
\[  FS =\langle CH, M_s, n_{{ch}_{le}} , n_{{ch}_{max}}, m_{ch}, tr, \]
\[TSO, SOF, FSO, FSPI \rangle,\] 
where
\begin{itemize}
\item $CH = \langle ch_1, ch_2, ..., ch_i, ..., ch_n \rangle, i =\overline{ 1, n}$ is the sequence of dynamically allocated file segments in sc-memory of fixed size n;
\item  $ch_i = \{\langle\langle l_{si1},s_{i1}\rangle, ...,\langle l_{sij1}, e_{ij} \rangle, ...,\langle l_{sim}, e_{im}\rangle\rangle,$ $n_{sl}, m_s\},j = \overline{1, m},$ — the i-the file segment of fixed size m, consisting of cells – pairs of string lengths $l_{sij}$ and stringsthemselves $sij \in ST R,$
\item  $n_{sl} \in \bar{\mathbb{N}}$ — the index of the last engaged cell in the file segment $ch_i$,
\item $m_s \in M$ — the object that synchronizes access to $n_{sle}$ ;
\item$ M_s \subseteq CHS \times M$ — a dynamic oriented set of file
and cell pairs and their corresponding synchronization objects;
\item  $n_ch_{le} \in \bar{\mathbb{N}}$ — the index of the last engaged file segment in sc-memory $(n_{chle} = n)$,
\end{itemize}
\end{multicols}
\end{SCn}
\end{document}