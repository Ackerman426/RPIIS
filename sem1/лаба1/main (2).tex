\documentclass[10pt]{article}
\usepackage{setspace}
\usepackage{titling}
\usepackage{multicol}
\usepackage{fancyhdr}
\usepackage{enumitem}
\usepackage{makeidx}
\usepackage{graphicx} % Required for inserting images
\setstretch{0.8}
\newcommand{\RomanNumeralCaps}[1]
    {\MakeUppercase{\romannumeral 1}}
\usepackage{geometry}
\geometry{}
\geometry{top=10mm}
\title{\textbf{Bringing the Subject Domain Ontology to
Optimal Canonical Form}} 
\date{}
\setlength{\droptitle}{-10pt}
\begin{document} 
\setcounter{page}{237}
\maketitle

\begin{center}
\vspace{-75pt}
    Anatoli Karpuk\\
Belarusian State Academy of Communications \\
Minsk, Belarus\\
Email: a\textunderscore karpuk@mail.ru\\
\end{center}
\begin{multicols}{2}
\textit{}
\textbf{\textit{Abstract}—A formal definition of the subject domain
ontology is given. The concept of the canonical form of
a subject domain ontology is considered, which is built on
the basis of an analysis of functional dependencies between
concepts and properties of the subject domain ontology
concepts. An algorithm for bringing the subject domain
ontology to a canonical form is described. The concept of
the optimal canonical form of a subject domain ontology
is introduced, containing the minimum number of classes
and the minimum number of attributes in the classes. A
method for bringing the subject domain ontology to the
optimal canonical form is roposed.\\
\textit{Keywords}—ontology, domain, canonical form, functional
dependence, optimal canonical form}
\begin{center}  
\RomanNumeralCaps{1}. Introduction
\end{center}
In computer science, ontology is a comprehensive and
detailed formalization of a certain area of knowledge in
the form of a conceptual diagram. A conceptual schema
is a set of concepts and information about concepts,
which includes properties, relationships, restrictions, axioms and statements about concepts necessary to describe
the processes of solving problems in a selected subject
domain. For each knowledge area, an applied ontology
is built, which consists of a top-level ontology, a subject
domain ontology, and a task ontology. Subject domain
ontology are simultaneously developed and used by
many users. For this reason, in subject domain ontology,
the same property of a concept can be represented in
different ways. Such ambiguity can lead to difficulties
when solving problems using subject domain ontology.
To eliminate this drawback, works [1], [2] propose to
bring the subject domain ontology to the so-called canonical form, which is based on the analysis of functional
dependencie of the
ontology. However, the developed algorithm for bringing
the ontology to a canonical form also does not provide
a unique solution. Depending on the order in which the
functional dependencies between attributes are analyzed,
it is possible to obtain a different number of classes
with different numbers of attributes in them. This article
introduces the concept of an optimal canonical form of a
subject domain ontology, containing a minimum number
of classes and a minimum number of attributes in classes,
and proposes a method for bringing a subject domain
ontology to an optimal canonical form.
\begin{center}
   \RomanNumeralCaps{1}\RomanNumeralCaps{1}. Formal Definition the Subject Domain Ontology
\end{center}
Let us define the subject domain ontology in the form
of a quadruple O = ⟨K, R, F, I⟩ [3], [4] where K is a
finite set of concepts of the subject domain ontology; R
– a finite set of relations between concepts; F – a finite
set of interpretation functions defined on concepts and
relationships; I – a finite set of axioms, each of which is
always a true statement on concepts and relations. The
set of concepts has the form K = ⟨D, A, Q⟩ , where D
is a finite set of domains; A — a finite set of attributes;
Q — a finite set of classes in subject domain ontology.

Domains are used as sets of possible attribute values.
Each domain’s data has one of the data types allowed
in the XML language [5]. Based on the number of data
elements in the value, domains are divided into atomic,
her aggregates and data elements. The
value of any merged domain can be represented as a
union of the values of its constituent atomic domains. A
list domain value is a list (repeating group) of atomic
or concatenated domain values. The number of list elements can be any. The value of any list domain can be
represented as a repeating group of values from one or
more atomic domains.

Based on value restrictions, atomic domains are divided into primitive, built-in, and constructed. Primitive
data types of the XML language are used as primitive atomic domains. Built-in beautiful very big domains are derived from
primitive domains by applying fixed constraints to them.
For example, the primitive domain decimal produces the
built-in domains integer, long, int, short, byte, nonNegativeInteger, positiveInteger, unsignedLong, unsignedInt,
unsignedShort, unsignedByte, nonPositiveInteger, negativeInteger. Derived domains are derived from primitive and built-in domains by applying various facets to
them. For example, the constraints length, minLength,
maxLength, pattern, enumeration, whiteSpace, assertions
can be applied to the primitive atomic domain string.The
\newpage 
\fancyhf{}
\fancyhead{+20pt}
 constraints totalDigits, fractionDigits, pattern, whiteSpace, enumeration, maxInclusive, maxExclusive, minInclusive, minExclusive, assertions can be applied to the
primitive atomic domain decimal.

Depending on the identification method, domains can
be unnamed or named. Unnamed domains do not carry
semantic load and are used only as data types. Unnamed
domains cannot be merged or list domains. Named
domains are distinguished from unnamed domains by
having a user-defined domain name and can be atomic
primitives, inline and derived domains, as well as federated and list domains.

Concepts from set A represent properties (attributes)
of subject domain classes. Each attribute is specified by
its unique name and the domain to which the attribute
values belong. Depending on what domain the attribute
is defined on, it can be atomic, aggregated, or list.

Concepts from set Q represent classes (objects, entities) of the subject area. One class includes real or
abstract people, objects, phenomena, events, processes
that have the same or similar set of properties (attributes),
knowledge about which is stored in the ontology and used
when solving problems from a given subject domain.
When constructing a subject domain ontology, classes are
first described, and then knowledge about the individuals
of each class is recorded in the ontology. In Russianlanguage literature, individuals of classes are often called
instances of classes or objects. Each class is given its own
unique name and its own set of attributes.

Each attribute within a class can have cardinality
and functional properties. Cardinality values indicate the
minimum and maximum number of attribute values that
one individual of a class can have. By default, a class
individual can have any number of attribute values. If
an attribute value may not be present in an individual
of a class, then the ontology must indicate a minimum
cardinality equal to 0. The functionality sign shows that
any individual of a class can have no more than one
attribute value. If the functionality attribute is set, then
the maximum cardinality of this attribute should not
be specified, or should be equal to 1. The set of class
attributes, the values of which uniquely determine an
individual of the class, is declared as the key of the class.
A class can have more than one key.

The set of relations between concepts R includes
relations between domains for constructing derived domains, relations between attributes and domains for
determining the scope of attributes, relations between
classes and attributes for determining the composition
of the attributes of each class, and relations between
classes. Relationships between classes reflect “wholepart”, “genus-type” connections, as well as hierarchical
and other connections between classes that exist in the
subject domain. Each relationship between classes can
also have cardinality and functionality properties. In
addition, relationships between classes can be inverse,
inverse functional, transitive, symmetric, asymmetric,
reflexive, and irreflexive.

The set of functions F consists of n-ary relations
between classes or attributes in which the value of an
element with a number n is uniquely determined by the
values of previous ($n-1$) elements. Using functions,
you can describe class keys, hierarchical relationships
between classes and attributes, and any other functional
dependencies between classes and attributes that exist in
the subject domain.

The set of axioms I serves to represent in the ontology statements about classes, attributes, domains and
relations that are always true. Each axiom is formulated
in the form “if <condition on the values of domains for
given attributes of given classes or relations> then <statement about the values of domains for given attributes
of given classes or relations>”. Axioms are included
in the ontology to check restrictions on the values of
attributes, to check the correctness of the description of
the ontology, to derive new true statements about classes,
attributes, domains and relationships.
\begin{center}  \RomanNumeralCaps{3}\RomanNumeralCaps{1}\RomanNumeralCaps{1}. Functional Dependencies between Attributes of the
Subject Domain Ontology
\end{center}

Let $X \subset A$ be a subset of attributes of the subject
domain ontology, $Z \in A$ — some attribute. We will say
that in the subject domain ontology there is a functional
dependence (FD) X → Z, if any combination of attribute
values from X always corresponds to a single value of
the attribute Z.

The FD structure on a set of attributes A satisfies
Armstrong’s axioms [6]:

if $X \subseteq A$ then A → X (reflexivity axiom);

if X → Y and Y C → D, then XC → D (axiom of
pseudotransitivity).

If there is a FD X → Y , then they say that X
functionally determines Y or Y functionally depends on
X. From the given axioms, one can derive a number of
properties of the FD structure, which in the literature (for
example, [7]) are often also called axioms, although it is
more accurate to call them rules of inference. The most
important are the following inference rules:

if X → Y C, then X → Y and X → C (decomposition rule), indeed, by the axiom of reflexivity we have
XY C → Y and XY C → C, then by the axiom of
pseudotransitivity we obtain X → Y and X → C;

if X → Y , then XC → Y C (replenishment rule),
indeed, by the axiom of reflexivity we have XY C →
Y C, then by the axiom of pseudotransitivity we obtain
XC → Y C;

if X → Y and X → C, then X → Y C (the union
rule), indeed, by the completion rule we have X → XY
and XY → Y C, then by the axiom of pseudotransitivity
we obtain X → Y C.
\newpage
Obviously, the inclusion relation $\subseteq$ determines the FD
structure on the set A, which is called the trivial structure
of the FD, and the FDs included in it are called trivial
FDs. To specify a FD structure that differs from the trivial
one, it is necessary to postulate a finite set of FD F =
{$F_j$ = $X_j$ → $Y_j$ | $X_j$ $\subset A$, $Y_j$ $\subseteq A$, j = 1, m}, which in
the article [8] was called a system of generators of the
FD structure on the set of attributes A. The FD structure,
specified by the system of generators F, will be denoted
by S(F).

It is obvious that from $Z \in X$ it follows that X → Z.
Such an FD, in which the dependent attribute is part of
the left side of the FD, is called trivial. In what follows,
we will consider only non-trivial FDs between attributes.
In the subject domain ontology, the following non-trivial
FDs between attributes can be distinguished:
\begin{itemize}[left=10pt,labelsep=0pt,itemsep=-6pt,topsep=0pt]
    \item  each functional attribute of a class that is 
        not a subclass of another class, that is not a subordinate attribute of another class attribute, functionally depends on each class key;
    \item each functional attribute of a subclass that is not a subordinate attribute of another attribute of a subclass is functionally dependent on each subset of attributes obtained by combining each key of the parent class with each key of the subclass;
    \item  each functional attribute of a subclass that is not
a subordinate attribute of another attribute of a
subclass is functionally dependent on each subset
of attributes obtained by combining each key of the
parent class with each key of the subclass;
    \item each functional attribute of a subclass that is not
a subordinate attribute of another attribute of a
subclass is functionally dependent on each subset
of attributes obtained by combining each key of the
parent class with each key of the subclass;
    \item each functional attribute of a class that is not a subclass of another class that is a subordinate attribute
of another class attribute functionally depends on
each subset of attributes obtained by combining
each class key with a parent attribute;
\item each functional attribute of a subclass, which is
a sub-attribute of another attribute of a subclass,
functionally depends on each subset of attributes
obtained by combining each key of the parent class
with each key of the subclass and the parent attribute;
\item  each functional relationship between classes from
the set R specifies the FD of attributes of each key
of the parent class from each key of the subordinate
class;
\item  each function from the set F, defined on the attributes of the subject domain ontology, sets the FD
of the last attribute of the relation from the previous
attributes of this relation;
\item  each function from the set F, defined on the classes
of the subject domain ontology, specifies the FD of
the attributes of each key of the last class of the
relation from each subset of attributes containing
any one key of the previous classes of this relation.
\end{itemize} 
When defining the FD between the attributes of the
subject domain ontology, it is possible to write more
than one attribute on the right side of the FD, since
for the FD between attributes the property of the cluster
decomposition of the right side is valid, namely, if
$Y_1 \subset A$, $Y_2 \in A$ and Y = $Y_1 \cup Y_2$,record
X → Y corresponds to the simultaneous presence of
FD X → $Y_1$ and X → $Y_2$. Moreover, instead of
the last two FD, you can write X → Y1Y2. Each FD
between attributes in the subject domain ontology can
be considered as the simplest rule for deriving new
knowledge from the knowledge available in the ontology.
Indeed, if the values of the attributes of the left side contains uniquely corresponding values of the 
of the right side of the FD, or the values of the attributes
of the right side of the FD can be obtained of this problem are the values of the attributes of the left
side of the FD, and the output data are the values of the
attributes of the right side of the FD.

Let us single out in the subject domain ontology
all FDs between attributes and represent them as a set
of FDs P = {$P_j$ = $X_j \to Y_j$ | $X_j \subset A$, $Y_j \subseteq \textbf{}A$, j = 1, m}, which is called the system of forming FD
structures on the set of attributes of the ontology. The
structure of the FD, given by the system of generators
P, will be denoted S(P). 

The closure of the set of attributes $X \subseteq  A$ concerning
to the structure of FD S(P) is a set $X+(P) \subseteq A$ such
that for any $Y \subseteq A$ from X → Y follows $Y \subseteq X+(P)$.
In other words, the closure of the set of attributes X
includes all the attributes, the values of which can be
obtained from the known values of the attributes of
set X, using the FD derivation from the set P. The
algorithm for constructing the closure X+(P) consists
of the following steps [8].
\begin{enumerate}[left=10pt,labelsep=0pt,itemsep=-5pt,topsep=0pt]
    \item  Put X+(P) = X and p\_j = 0, j = 1, m.
    \item Put q = 0 and for each j = 1, m perform step 
          3.
    \item If p\_j = 0 and $X\_j \subseteq X+(P)$ then put X+(P) =
             $X+(P) \subset Y\_j$ , q = 1 and p\_j = 1.
    \item  If q = 1, then go to step 2, otherwise finish the job.
\end{enumerate}
The structures of FD S(P1) and S(P2) on the set of
attributes A with systems of generators P1 = {$X_i^1\to Y_i^1$ | $X_i^1 \subset A$, $Y_i^1 \subseteq A$, i = 1, m_1} and P^2 ={$X_j^2 \to Y_j^2$ | $\X_j^2 \subset A$, $Y_j^2 \subseteq A$, j = 1, m\_2} ,
respectively, are called equivalent if for any $X \subset A$
the equality $X+(P^1)$ = $X+(P^2)$. In article [9] it
is proven that necessary and sufficient conditions for
the equivalence of structures FD S(P1) and S(P2) on
the set of attributes A are the fulfillment of equalities
$X_i^{1+(P^1)}$ = $X_i^{1+(P^2)}$ and $X_j^{2+(P^1)}$ = $X_j^{2+(P^2)}$
for all i = 1, $m_1$, j = 1, $m_2$. The system of generators
E = {$H_j \to T_j$ | $H_j \subset A$,$T_j\subseteq $, j = 1, m}
is called
an elementary basis of the structure of FD S(E) if the
removal of any attribute from the left or right side of
any FD from E leads to the structure of FD that is not
equivalent to S(E).


\end{multicols}


\end{document}
