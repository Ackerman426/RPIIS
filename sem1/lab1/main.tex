\documentclass[twocolumn]{scndocument}
\usepackage{import}
\usepackage{scn}
\usepackage{fancyhdr}
\usepackage{graphicx}
\pagestyle{fancy}
\fancyhf{}
\fancyfoot[C]{\textbf{\thepage}}
\DeactivateBG
\usepackage[version=4]{mhchem}
\renewcommand{\theenumi}{\arabic{enumi})} 
\renewcommand{\labelenumi}{\theenumi} 


\begin{document}

\setcounter{page}{150}

The current state of art of existing medical decision support systems is presented in [4]–[7] and demonstrates that almost every system is focused on a specific disease or group of diseases.

To date, a large number of private, highly specialized decision support systems have been developed. For
example, SkyChain is designed for diagnosing lung,
liver, breast, and melanoma cancer. The IDDAP system
identifies potential infectious diseases and disease states
based on the constructed ontology of the subject area.
MYCIN is an interactive expert system for diagnosis and
treatment of infectious diseases [8]. In [9], an intelligent
system for personalized human health monitoring based
on biomedical signal processing is designed using the
Internet of Things, cloud computing, big data processing
and neural network.

The number of localized problem statements and their
solutions is so large that it is almost impossible to catalog
them and have a common source of information about
them. Localized solutions can be found in various information and search engines: PubMed, Scopus, Google
Scholar, WoS.

It is difficult for medical staff to use several systems at
once in practice, and it is expensive to develop and maintain such systems. Most of the existing systems focus on
diagnostics on late stages of disease, while virtually none
of the existing system considers the diagnosis in the early
stages of the disease.

In terms of diagnostic methods, this article focuses on
non-invasive methods of examination for early stages of
the disease. Since they are highly informative, do not
require long additional preparation of the patient, significant time expenditures, and also during the procedure
the integrity of the skin is preserved.

\begin{center}
\vspace{-3pt}
III. Problem Statement
\vspace{-3pt}
\end{center}

The design and implementation of intelligent noninvasive diagnostic system will require the consideration
of identification of technological platform capable to
process and support the non-invasive diagnostics. The
methodological approach should address the solution of
at least three steps:

The first step is to investigate promising methodological directions (variants) of non-invasive signals suitable
for implementation of intelligent diagnostics: FunctionalSpectral Diagnostics (FSD-diagnostics); preliminary diagnostics based on the assessment of basic parameters
of functional state (such as electrocardiogram (ECG),
arterial blood pressure, heart rate (pulse, HR), temperature distribution in the local skin area, carbon dioxide
volume in exhaled air CO2, arterial blood oxygen saturation SpO2); bioimpedance analysis; Zakharyin-Ged zone
diagnostics; Nakatani method diagnostics; frequencyresonance diagnostics.

The second step is to formulate general requirements
to the diagnostic decision support system.

The Decision Support System for Diagnostics is designed for doctors and patients.

When developing a decision support system for human
diagnostics, we believe it is reasonable to shift the
focus of attention not only to the development of an
intelligent system, but also to the issues of integration
and compatibility of different solutions and approaches,
and their subsequent joint development.

When developing a decision support system for human
diagnostics, we believe it is reasonable to shift the
focus of attention not only to the development of an
intelligent system, but also to the issues of integration
and compatibility of different solutions and approaches,
and their subsequent joint development.

The third step is to justify the choice of technology
for the development of an intelligent system.

Following the formulation of methodological approach, the technological requirements of the system can
be summarized as following. The proposed intelligent
non-invasive diagnostics system should:

\vspace{-5pt}
\begin{itemize}
    \setlength{\itemsep}{-3pt} 
    \setlength{\parsep}{-3pt}
    \item be oriented to the design and development (improvement) of the system;
    \item provide the possibility of integration of heterogeneous data pipelines:
    \item process different types of data
    \item ensure standardization of data representations
(forms of representation, information processing
models);
    \item support modular system architecture that provides
the possibility of adding new components and new
data types, and their integration into the system;
    \item ensure integration with modern emerging technologies.
    \item ensure compatibility of interfacing with different
systems, their docking.
\end{itemize}
\vspace{-5pt}

The choice of technology to be developed should
include the possibility of strengthening technological
sovereignty. Since full technological dependence of countries on monopolistic corporations, which we are now
witnessing, is a "path dependence" in the current situation. One of the solutions is the development of opensource projects and focusing on efficient exploitation of
open-source libraries openly available for research and
industrial use free of charge.

\begin{center}
\vspace{-3pt}
IV. Proposed Approach
\vspace{-3pt}
\end{center}

The purpose of the system is to assist physicians in
establishing a diagnosis. For this purpose, it is necessary
to establish in the system the principles of diagnosis
formation, the structure of the diagnosis, as well as
to provide a transparent mechanism of reasoning when
making a diagnosis or when proposing several diagnostic
hypotheses. Knowledge driven systems solve these problems.

To define a technical solution it is necessary to take
into account the three key elements: (1) comprehensibility of the reasoning process, (2) heterogeneous data
representation, and data and (3) interface standardisation.

Firstly, when designing decision support systems in
medicine, it is necessary to take into account that for
practical application these systems require explainability of the reasoning process with provision of reasoning
on how the results obtained, as well as the ability to
modify the knowledge used in the system in a timely
manner. Fulfillment of such requirements is ensured by
the ontological approach [10], on the basis of which
knowledge bases (KB) in terms and structure familiar
to specialists can be created.

Secondly, since different types of data are used as
input data, it is necessary to integrate heterogeneous data
pipelines, which is ensured by the ontological approach.

Thirdly, when building decision support systems, one
of the problems is to provide a uniform description and
interpretation of data, regardless of the place and time
of their receipt in the overall system. One of the ways to
solve this problem can be the introduction of ontological
modeling technologies.

A number of other advantages of the ontological
approach should be emphasized. Ontologies use logical
formalisms, which makes them convenient for use in the
development of complex systems. Ontologies are characterized by flexibility, which allows combining information from different sources and building new knowledge
on its basis. The main purpose of creating any ontology
is to model a certain subject area, in turn, this forms
the core of the system, and other modules are easily
integrated with this module. Also the knowledge stored
in ontological form has a high potential for reuse. Each
ontology contains some fragment of conceptual knowledge of the subject area and hence ontology systems are
called knowledge-based systems.

A diagnosis ontology includes a structure for describing information, rules for interpreting it and applying it
to diagnosis.

In view of the above, it is reasonable to use the
ontological approach.

Logical-semantic systems are based on ontologies.
Artificial intelligence systems based on logical-semantic
knowledge processing work with conceptual apparatus.
Logical-semantic systems work with knowledge representations in the form of ontologies, realized, for example, in the form of a knowledge graph, where concepts
and other objects correspond to the nodes of the graph
and relations between them — to the edges of the graph.

As applied to the tasks of non-invasive diagnostics
logical-semantic systems will allow to build parallel
hypotheses and form a diagnosis (more or less probable),
which is an advantage of their use.

Alternative development tools include ontology editors
Protégé, Ontolingua, OntoEdit and others. Recently, the
number of publicly available ontology editors has been
increased. However, the main problem is that ontology
editors are considered in isolation from implementation
technology.

The most effective technology for creating ontological
systems is the domestic technology of complex life cycle support of semantically compatible intelligent computer
systems of new generation (Open Semantic Technology
of Intelligent Systems — OSTIS) [11], [12].

OSTIS technology is an open semantic technology
for component-based design of hybrid and interoperable
intelligent systems.

The OSTIS technology is based on the OSTIS standard, which is a standard of semantic computer systems
that provides

\vspace{-5pt}
\begin{itemize}
    \setlength{\itemsep}{-3pt} 
    \setlength{\parsep}{-3pt}
    \item semantic compatibility of systems complying with
this standard;
    \item methods of building such computer systems and
their improvement in the process of operation;
    \item means of building and improving these systems,
including language tools, libraries of standard technical solutions, as well as tools (means of synthesis
and modification; means of analysis, verification,
diagnosis, testing; means of [10].
\end{itemize}
\vspace{-5pt}

It should be noted that one of the most important
features of systems built on the basis of OSTIS technology is their platform independence. It has an orientation
on the semantic representation of knowledge, which
is completely abstracted from the peculiarities of the
technical realization of intelligent systems.

It is important to consider the possibility to use already
developed ontologies on medicine (on various resources),
pre-transforming them into OSTIS format with further
processing of the expert (adaptation to specific tasks).
For example, it is possible to use "Knowledge bases
of medical terminology and observations", ontology for
representing knowledge about diagnostics of diseases and
syndromes realized on the cloud platform IACPaaS [13]
and others.

\begin{center}
\vspace{-3pt}
V. Description of the system operation principle
\vspace{-3pt}
\end{center}

Problem Statement. We aim design the architecture
of medical decision support system for patient diagnosis
based on non-invasive diagnostics.

Tools for realization — OSTIS technology.

The architecture of the OSTIS system for medical
applications consists of the following components:

\vspace{-5pt}
\begin{itemize}
    \setlength{\itemsep}{-3pt} 
    \setlength{\parsep}{-3pt}
    \item OSTIS Knowledge Bases — Can describe any type
of knowledge, while being easily augmented with
new types of knowledge. Can include such ontologies as disease model, etc.
    \item OSTIS problem solver is based on multi-agent approach and allows easy integration and combination
of any problem-solving models.
    \item The OSTIS interface is a subsystem with its
own knowledge-base and problem solver (separate
knowledge base and problem solver).
\end{itemize}
\vspace{-5pt}

The intelligent diagnostic decision support system will
include the following components:

\vspace{-5pt}
\begin{itemize}
    \setlength{\itemsep}{-3pt} 
    \setlength{\parsep}{-3pt}
    \item Block of input data collection and storage
    \item Data processing unit
    \item Block of diagnostic conclusions formation (mechanism of logical conclusion, mechanism of "reasoning" when making a diagnosis, allowing to get
an idea of what information was the basis for the
diagnosis).
    \item Block of consulting doctors and patient — issuing
a response to the end user of the system about the
results of diagnostics, recommendation on further
actions.
\end{itemize}
\vspace{-5pt}

\begin{center}
\vspace{-3pt}
VI. Non-Invasive Diagnostic Methods
\vspace{-3pt}
\end{center}

The term "non-invasive" can be translated as "without
disturbing the skin".

The following criteria are taken into account when for
choosing a non-invasive diagnostic method:

\vspace{-5pt}
\begin{itemize}
    \setlength{\itemsep}{-3pt} 
    \setlength{\parsep}{-3pt}
    \item simplicity of the procedure;
    \item safety and painlessness for the person;
    \item examination time;
    \item the number of features (e. g., markers);
    \item the number of body systems covered.
\end{itemize}
\vspace{-5pt}

Intelligent diagnostics of the human condition is focused on a set of data and considers the human body as
a whole.

Non-invasive human diagnostics can be carried out in
several directions (variants) of functional diagnostics.

\vspace{-5pt}
\begin{enumerate}
    \setlength{\itemsep}{-3pt} 
    \setlength{\parsep}{-3pt}
    \item  Functional spectral-dynamic diagnostics (FSDdiagnostics) for early diagnosis of diseases [14].
    \item  Bioimpedance analysis [15].
    \item Preliminary diagnosis based on the evaluation of
basic parameters of the functional state, such as
electrocardiogram (ECG), arterial blood pressure,
heart rate (pulse, HR), temperature distribution in
the local skin area, volume of carbon dioxide in
exhaled air \textit{\ce{CO2}}, arterial blood oxygen saturation
\textit{\ce{SpO2}}. These measurements can be carried out, for
example, using the device "Patient Monitor" [2],
[16].
    \item Diagnosis by Zakharyin-Ged zones [17].
    \item Diagnosis by the Nakatani method [18].
    \item Frequency-resonance diagnostics (bioresonance).
\end{enumerate}
\vspace{-5pt}

The choice of features mentioned above is justified by
their prevalence and technological efficiency.


For each identified feature the principle of passivity
of the main mode of diagnostics (without influence on
the organism) or the principle of activity of the mode
of diagnostics (influence on the organism is present)
becomes an important indicator

The passive indicators include: FSD-diagnostics;
Zakharyin-Ged zone diagnostics; measurement of electrocardiogram (ECG), arterial blood pressure, heart rate
(pulse, HR), temperature distribution in the local area of
the skin, volume of carbon dioxide in exhaled air \textit{\ce{CO2}},
arterial blood oxygen saturation \textit{\ce{SpO2}}.

The active indicators include: bioimpedance analysis,
Nakatani method diagnostics, frequency-resonance diagnostics.

Let’s consider the in great detail non-invasive diagnostic methods mentioned above.

\begin{flushleft}
\vspace{-3pt}
\textit{A. FSD-diagnostics}
\vspace{-3pt}
\end{flushleft}

The most effective diagnostic technology is the technology of functional spectral-dynamic diagnostics (FSDdiagnostics) applied to solve the tasks of health dynamics monitoring [14]. This is due to the fact that
FSD-diagnostics is effective with respect to common
infectious and non-infectious diseases, including latent
(hidden) stages and actual risks of their development.
FSD provides a priori sufficiency (due to markers).

The FSD diagnostic technology is focused on the
detection of disease risks (that is often used during early
diagnosis). The sensor is a metal electrode that records
a wave electromagnetic signal in the sound range.

The core principle of the spectral-dynamic method is
to analyze the electrical oscillations of the body field in
the frequency range from 20 hertz to 11 kilohertz with
an amplitude of 1 millivolt.

FSD diagnosis involves the following operations [19]:

\vspace{-5pt}
\begin{itemize}
    \setlength{\itemsep}{-3pt} 
    \setlength{\parsep}{-3pt}
    \item  tool: a sensor (metal electrode) is applied to the
patient’s skin surface for 35 sec;
    \item  data collection: recording of the body wave signal
in the frequency range from 20 Hz to 11 kHz (EMF
audio/\+-\--+/*[-range) is performed;
    \item signal processing: spectral analysis of the signal
based on Dobeshi wavelet transform 3;
    \item detection problem: recognizing the presence of
spectral correspondences with similar spectra of
electronic copies of reference diagnostic markers;
    \item spectral correspondence (similarity of the marker
with the corresponding part of the patient’s spectrum) expressed in percent is the main indicator for
the physician who issues a diagnostic report;
    \item the diagnostic report contains indications of risks or
presence of infectious and non-infectious diseases.
\end{itemize}
\vspace{-5pt}

Distinctive features of FSD-diagnostics from existing
diagnostic technologies are: the principle of pattern
recognition instead of the principle of parameter measurement; the principle of passivity of the main mode of
diagnostics (without impact on the body); the possibility
of automation of nosological diagnostics (recognition of
the disease itself or nosological risk). The examination
is performed in less than one minute, and its diagnostic
analysis can take up to two hours.

\begin{flushleft}
\vspace{-3pt}
\textit{B. Bioimpedance analysis (bioimpedanceometry)}
\vspace{-3pt}
\end{flushleft}

Bioimpedance analysis (from "biological" and 
"impedance" — complex electrical resistance,
"bioimpedance" — electrical resistance of biological
tissues) — analysis of the amount of fat and fluid in
the body, muscle and bone mass and metabolism, a
method of rapid diagnosis of human body composition
by measuring the electrical resistance between different
points on the human skin.

 
\end{document}
