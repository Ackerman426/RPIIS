\documentclass[10pt, a4paper]{article}
\usepackage{hyperref}
    \usepackage[unicode, pdftex]{hyperref}
\usepackage[russian, english]{babel} % установка языков
\usepackage{graphicx} % вставка изображений
\usepackage{multicol} % деление на колонки
\usepackage{fancyhdr} % настройка колонтитулов
\usepackage{parskip} % настройка отступов абзаца
\usepackage{biblatex} % список литературы
\usepackage{mathtext} % Times New Roman
\usepackage{indentfirst} % отступ после заголовка секции
\usepackage{float} % плавающие картинки
\usepackage{titlesec} % настройка заголовков
\usepackage[margin=0.1cm]{caption} % настройка описаний
\captionsetup[figure]{font=scriptsize} % шрифт описания фигуры
\usepackage[left=2.5cm,right=2.5cm, top=2.1cm,bottom=2.5cm]{geometry} % макет страницы
\fancyhf{} % очистка колонтитулов
\cfoot{\textbf{\thepage}} % жирные номера страниц
\pagestyle{fancy}
\renewcommand{\headrulewidth}{0pt} % удаление линии header
\linespread{0.99} % межстрочный интервал
\setlength{\columnsep}{0.6cm} % расстояние между столбцами

\setlength{\parskip}{0pt} % вертикальный отступ абзаца
\setlength{\parindent}{0.5cm} % горизонтальный отступ абзаца
\setlength{\textfloatsep}{\theintvl\curtextsize} % отступ после картинок
\setcounter{page}{108} % нумерация страниц
\setcounter{figure}{7} % начальное значение нумерации фигур
\titleformat{\section}{\normalsize\centering}{\thesection. }{0cm}{}[] % стиль заголовков разделов
\titlespacing*{\section} % отступ возле секций
{0pt}{0.4cm}{0.3cm}
\renewcommand{\thesection}{\Roman{section}} % римские цифры
\setcounter{section}{6} % нумерация секций

\begin{document}


\begin{figure}[H]
    \href{http://fkn.ktu10.com/?q=node/2906}{%
        \parbox{\textwidth}{
            \centering
            \includegraphics[width=16cm]{lw1_picture1.jpg}
            \caption{Methodology of machine learning model development for computer vision}
            \label{fig_8}
        }
    }
\end{figure}


\begin{multicols}{2}
\noindent in the high complexity of their maintenance and improvement, as well as their insufficiently long life cycle.


The problems of the unification of the principles for
constructing various components of computer systems
are solved in the OSTIS project. The OSTIS The project
aims to create an open semantic technology for designing knowledge-driven systems in general and computer
vision systems in particular [13].
\section{Conclusion}

The methodology for developing a machine learning
model for solving applied computer vision problems is
presented. The article discusses the tasks of computer
vision, the main components of building application
systems, and the and the challenges and limitations of
the existing technological level. demonstrated the need
to develop areas related to the compatibility of scientific
research results in the field of artificial intelligence.
\begin{footnotesize}
\begin{thebibliography}{13}
    \bibitem{book1} Karen Simonyan and Andrew Zisserman. Very deep convolutional
networks for large-scale image recognition. In ICLR, 2015.
    \bibitem{book2} Alex Krizhevsky, Ilya Sutskever, and Geoffrey E Hinton. Imagenet
classification with deep convolutional neural networks. Communications of the ACM, 60(6):84–90, 2017.
    \bibitem{book3}  Sebastian Raschka; Yuxi (Hayden) Liu; Vahid Mirjalili; Dmytro
Dzhulgakov, Machine Learning with PyTorch and Scikit-Learn:
Develop machine learning and deep learning models with Python
, Packt Publishing, 2022.
    \bibitem{book4} Fermüller, Cornelia and Michael Maynord. “Advanced Methods
and Deep Learning in Computer Vision.” (2022).
    \bibitem{book5} Zhengxia Zou, Keyan Chen, Zhenwei Shi, Yuhong Guo, and
Jieping Ye. Object detection in 20 years: A survey. Proceedings
of the IEEE, 2023.
    \bibitem{book6} Tianfei Zhou, Fatih Porikli, David J Crandall, Luc Van Gool, and
Wenguan Wang. A survey on deep learning technique for video
segmentation. TPAMI, 2023.
    \bibitem{book7}  Jingjing Xu, Wangchunshu Zhou, Zhiyi Fu, Hao Zhou, and
Lei Li. A survey on green deep learning. arXiv preprint
arXiv:2111.05193, 2021.
    \bibitem{book8}  Jianyuan Guo, Kai Han, Han Wu, Yehui Tang, Xinghao Chen,
Yunhe Wang, and Chang Xu. Cmt: Convolutional neural networks
meet vision transformers. In CVPR, pages 12175–12185, 2022.
    \bibitem{book9}  R. K. Kumar and K. Arunabhaskar, "A Hybrid Machine Learning
Strategy Assisted Diabetic Retinopathy Detection based on Retinal Images," 2021 International Conference on Innovative Computing, Intelligent Communication and Smart Electrical Systems
(ICSES), 2021, pp. 1-6, doi: 10.1109/ICSES52305.2021.9633875.
    \bibitem{book10}  Dagliati, A, Marini, S, Sacchi, L, Cogni, G, Teliti, M, Tibollo,
V, De Cata, P, Chiovato, L & Bellazzi, R 2018, ’Machine
Learning Methods to Predict Diabetes Complications’, Journal
of Diabetes Science and Technology, vol. 12, no. 2, pp. 295-302.
https://doi.org/10.1177/1932296817706375
    \bibitem{book11}  Lukashevich M.M. A neural network classifier for detecting diabetic retinopathy from retinal images. «System analysis
and applied information science». 2023;(1):25-34. (In Russ.)
ttps://doi.org/10.21122/2309-4923-2023-1-25-34
    \bibitem{book12}  Labrak, Y., Rouvier, M., Dufour, R. (2023). A zero-shot and
few-shot study of instruction-finetuned large language models applied to clinical and biomedical tasks. arXiv preprint
arXiv:2307.12114.
    \bibitem{book13}  V. Golenkov, N. Guliakina, V. Golovko, V. Krasnoproshin,
“Methodological problems of the current state of works in
the field of artificial intelligence,” Otkrytye semanticheskie
tekhnologii proektirovaniya intellektual’nykh system [Open semantic technologies for intelligent systems], pp. 17–24, 2021.
\end{thebibliography}
\end{footnotesize}
\begin{otherlanguage}{russian}
\begin{center}
\vspace{0.3cm}
\small{\textbf{МЕТОДОЛОГИЯ РАЗРАБОТКИ МОДЕЛЕЙ МАШИННОГО ОБУЧЕНИЯ ДЛЯ РЕШЕНИЯ ПРИКЛАДНЫХ ЗАДАЧ КОМПЬЮТЕРНОГО ЗРЕНИЯ}}
\vspace{0.1cm}
\large{Лукашевич М. М.}
\end{center}

Представлена методология разработки моделей машинного обучения для решения прикладных задач
компьютерного зрения. В статье рассматриваются задачи компьютерного зрения, основные компоненты
построения прикладных систем, проблемы и ограничения существующего технологического уровня.
\end{otherlanguage}
\begin{flushright}
Received 13.03.2024
\end{flushright}
\end{multicols}
\begin{center}
\LARGE{\textbf{Principles and Experience of Intelligent Decision
Support and Recommender Systems Engineering}}
\end{center}
\vspace{0.4cm}
\begin{figure}[H]
\begin{minipage}{7cm}
\centering
    Boris Zhalezka \\
\textit{Department of Marketin \\
Belarusian National Technical University} \\
Minsk, Belarus \\
boriszh@yandex.ru
\end{minipage}
\begin{minipage}{9cm}
\centering
    Volha Siniauskaya \\
\textit{Department of Industrial Marketing and Communications\\
Belarusian State Economic University} \\
Minsk, Belarus \\
volha.siniauskaya@outlook.com \\
\end{minipage}
\end{figure}
\begin{multicols}{2}


    \textbf{\textit{\begin{otherlanguage}{russian}Аннотация\end{otherlanguage}}—In the article main principles of engineering
of intelligent decision support and recommender systems
are considered. Definition and concept of a generalized
object are formulated. Technologies of recommender systems
engineering and construction are analyzed. Classification of
decision support systems is suggested. Experience of decision
support and recommender systems engineering is presented.}

    \textbf{\textit{Keywords}–OSTIS, intelligent system with integrated spatially referenced data, semantic model, question language,
design process automation}
\setcounter{section}{0}
\section{Introduction}

Despite existing developments in the area of decision
support systems (further — DSS) engineering, these
technologies presuppose the adaptation of only individual
components of the DSS and do not provide the adaptation
of the subject area model. This leads to the use of irrelevant
and inaccurate data in the DSS, which negatively affects
the efficiency of decision-making necessary in a quickly changing
environment. These problems can be solved via adapting
subject known area models to the conditions of decision-making
tasks and well-timed updating of the models with the data,
knowledge and precedent (subject) collections necessary
for this model. The concept of a generalized object
became the base of different DSS with combined intellect
engineering and construction. The target of this research
is to generalize from theoretical and practical experience in the
sphere of intelligent decision support and recommender
systems construction.

The target of this research is to generalize theoretical and
practical experience in the sphere of intelligent decision
support and recommender systems construction.
\section{Brief literature review}
First theoretical research on decision support systems
was made in USA at the Carnegie Institute of Technology
in the late 1950s — early 1960s. The first main works on
DSS were published in 1978-1980 by P.Keen, M. Scott
Morton [1], [2], R. Sprague [3].

Investigations in the sphere of intellectualizing of
DSS information technologies are wavy carried out in
the world. Moreover, the increase in research activity
in time coincides with the periods of development on
computing and financial resources (the emergence of
personal computers, evolution of the processors, memory
capacity, the emergence of a user-friendly interface, the
emergence of mobile Internet technologies, etc., with the
simultaneous reducing of its prices).

Among the modern works on DSS there are
books “Intelligent Decision Support System for IoTEnabling Technologies: Opportunities, Challenges and
Applications” (2024) by ed. S. Sahana [4]; “Intelligent
Decision Support Systems for Smart City Applications”
(2022) by L. Gaur, V. Agarwal and P. Chatterjee [5];
“Intelligent Decision Support Systems” (2022) by M.
Sanchez-Marr ` e [6]; “Understanding Semantics-Based `
Decision Support” (2021) by S. Jain [7]; “Intelligent
Decision Support Systems: A Complete Guide — 2020
Edition” by G. Blokdyk [8].

The technology of recommender systems gained
popularity only in the middle of 1990s. The concept
of a recommender system was first used in 1992
in the scientific publication of Xerox, in the same
year in the article “Using collaborative filtering to
weave an information tapestry” the term “collaborative
filtering” was introduced. Subsequently, fundamental
works systematizing knowledge on recommender systems
were devoted to this area. One of them is “Recommender
Systems: The Textbook” [9]: in this book, description,
comparison, assessment of the accuracy of basic
algorithms for developing recommendations to the user
were made, in addition, the field of practical use of
such systems is affected. The work “Recommender
Systems: The Handbook” [10] deserves special attention.
In it the existing variety of methods and concepts of
recommender systems was systematized. This source
shows how recommender systems can help users in
decision-making, planning and procurement processes,
illustrates the experience of using these systems in big
corporations such as Amazon, Google, Microsoft.

Nowadays the transition to the next-generation
computer systems takes place. Intelligent DSS and
recommender systems belong to this class of information
\end{multicols}

\begin{multicols}{2}
\noindent systems. Such systems should “independently evolve and
interact effectively with each other in the collective
solution of complex problems” [11]. One of the relevant
problems of next-generation computer systems design
and development is generalization of formal theory and
methodology of their functioning.

\section{Definition and the concept of objects}
Usually an object O can be represented in the view
O $=<$ Data, Met, Mes $>$, where: Data — a set of
internal information of the object (data); Met — a set of
its own procedures for manipulating the data (methods);
Mes — an external interface for interacting with other
objects in the subject area (such as a permissible set of
event messages outside and inside of the subject area).

However, the need to take into account the development
of DSS requires a more general and flexible mechanism
for describing and modeling them. Such a mechanism can
be built on the base of further generalization of the objectoriented approach and, in particular, of the term “object”
in the conceptual model of the subject area

The term “generalized object” is suggested: GO =<
Data, Met, Model, Knowl, Mes, Link >, in which
models, knowledge and links with the other domain
objects are encapsulated in addition to data, methods and
messages

Such model of a generalized subject area can be
considered as a type of multi-object neural network, if
add to the usual method of object interaction by messages
the possibility of activating objects (transmission of
excitation) through (priorities)
indicating the value of the response threshold level for
each object. The value of the response threshold level
can with the system development
stage, solved problems, accumulated statistical experience
of problems solving, etc. Interaction between generalized
objects can be carried messages or changes in
the links structure of the generalized object system. This
system is separated from their functional part. It can be
and represented by a dynamic links list, by changes in the
links weights (filtering) and actuation threshold values.

Each generalized object can have several states and go
from one state to another depending on the incoming
messages that are the result of the activities of other
generalized objects. At the same time, the generalized
object changes its state when its excitation value exceeds
some non-zero threshold of actuation. In general, such
subject area model can be considered as a hierarchy of
abstractions classes of,
problem and user objects. The status of the subject area
actually depends on the status of each generalized object
and the message queues at the input and output of these
generalized objects. The latter can be considered as a
database of facts about events on the base of which it is
possible to determine an output machine for interpreting of existing and generating new facts in the process of
simulating of modelling system functioning.

Requirements to the multi-agent DSS can be formulated
by means of object-oriented classification (enumerations
of the object classes involved in solving problems, their
properties, relationships and behavior). The main principle
of classification is the of object classes based
on the set of internal properties inherent in class objects.
After that, the requirements are sequentially detailed until
the multi-agent DSS project is fully described in terms of
the basic objects of the used tools.

\section{Recommender systems}
In today’s world, we can often face the problem of
recommending goods and services to users of any site,
information system. The modern economic formation
involves intense competition in various market niches,
which leads to each potential buyer among
companies. In order to improve their experience of
interacting with the company’s services, it is beneficial
to create personalized collections that will have client
response. Previously, list, a
set of current actions and the most popular goods was
sufficient, but the current situation does not allow such
low-cost methods to act. A relatively new technology of
recommender systems can be used for buyers’ attraction
and sales increasing.

The essence of recommender systems approach
presupposes the dynamic formation of recommendations
personally for each specific client, which, unlike
static information, significantly increases probability of
coincidence with the real needs of people. Recommender
systems take into account all kinds of parameters (purchase
history, time and date of registration, region, purchased
products, etc.), which allows prediction of the user’s
wishes as accurately as possible.

Recommender systems have already found their place
in many areas: in addition to e-commerce, this kind of
technology has been introduced for finding books, films,
music, and social media contacts.

The relevance of recommender systems is growing
every year. Recommender systems are the programs aimed
for the prediction of the user’s interest in certain objects
and giving them the recommendation for purchase or
using the items, which the user probably likes. Such
recommendations are personalized and are formed for
each user depending on their preferences. Although
the technology itself appeared quite recently, its use is
considered mandatory for all promising companies.

One of the first to become interested in introducing
recommender systems was the largest American ecommerce platform Amazon. Already in the late 1990s,
the best minds of the company developed their own
algorithms for the so-called collaborative (joint) filtering,
which offered recommendations to each client based on
\end{multicols}
\end{document}
