\documentclass[10pt,two column]{article}
\usepackage{scn}
\usepackage{CJKutf8}
\usepackage[utf8]{inputenc}
\usepackage{multicol} %колонки
\usepackage{setspace} %межстрочный интервал
\usepackage{ragged2e}% выравнивания текста по ширине в документе.
\usepackage{fancyhdr} %настройки верхнего и нижнего колонтитулов в документе.
\usepackage{indentfirst} %чтобы была красная строка
\usepackage{titlesec} %стилей заголовков разделов в документе.
\usepackage{enumitem} %настройки списков в документе.
\usepackage{graphicx}%Вставка картинок правильная
\usepackage{float}%"Плавающие" картинки
\usepackage{wrapfig}%Обтекание фигур (таблиц, картинок и прочего)
\usepackage{amssymb}%для вставки символов
\usepackage[left=1.9cm,right=1.9cm, top=1.1cm,bottom=2.5cm]{geometry}

\justifying % выравнивает текст по ширине.
\fancyhf{} %очищает все верхние и нижние колонтитулы.
\renewcommand{\headrulewidth}{0pt} % remove the header rule
\cfoot{\vskip -1.5cm \thepage} %устанавливает номер страницы в нижнем колонтитуле.

\linespread{0.84} %устанавливает межстрочный интервал в 0.84.


\setlength{\columnsep}{0.5cm}


\setcounter{page}{249}
\renewcommand{\thesection}{\textbf{\Roman{section}}} %устанавливает стиль нумерации разделов в виде заглавных римских цифр.

\titleformat{\section}{\centering\sc}{\thesection.}{0cm}{}[] %настраивает стиль заголовков разделов.

\title{ \huge \textbf{ IoT Network for Diagnosis of Parkinson’s
disease Using Neural Networks and OSTIS } }
\newcommand{\SCnSpecialChar}{\textsf{C\kern-0.1emS}}


\begin{multicols}{2}
\author{
    Vishniakou Uladzimir \\
    \small{ \textit{Belarusian State University of} }\\
    \small{ \textit{Informatics and Radioelectronics} }\\
    \small{ Minsk, Belarus }\\
    \small{ vish@bsuir.by }
    \and
    Yiwei Xia \\
    \small{ \textit{Belarusian State University of} }\\
    \small{ \textit{Informatics and Radioelectronics} }\\
    \small{ Minsk, Belarus } \\
    \small{ xiayiwei4@gmail.com }
}
\end{multicols}
\date{}



\begin{document}

\maketitle 

\setlength{\parindent}{0.4cm}
\small{
\textbf{Abstract—The aim of the work is to propose a model for the diagnosis of Parkinson’s disease (PD) within the
framework of the Internet of Things (IoT) using OSTIS.
The report is devoted to the development of the Internet
of Things for the diagnosis of Parkinson’s disease (PD)
using OSTIS technology. The structure of the ontology
for describing the elements of PD disease is given. The
construction of an IT diagnostic network of the PD is
considered, which uses the semantic capabilities of the
OSTIS platform for processing and analyzing medical data
of the PD. The elements of the description of the knowledge
base, solvers and user interfaces for PD using a componentbased approach are presented.
}

\textbf{ \textit{Keywords}—Internet of Things network, IT diagnostics
of Parkinson’s disease (PD), PD ontology, neural networks,
knowledge base, OSTIS
}
}
\singlespacing
\section{ \textbf{ DIAGNOSIS METHOD APPLIED TO
PARKINSON’S DISEASE} } 

The early diagnosis of Parkinson’s disease has been a
challenge for the medical community, and usually about
60 \% of nigrostriatal neurons have degenerated and 80 \%
of striatal dopamine is depleted by the time the disease
is diagnosed [1]. Currently, the diagnosis of the disease
is based on medical history, clinical signs and symptoms,
and response to antiparkinsonian drugs, but because the
disease starts slowly and clinical symptoms appear only
when the nigrostriatal dopamine neurons are depleted to
a certain extent, patients are often at an advanced stage
of the disease when they are diagnosed, missing the best
time for treatment. Therefore, the development of new
treatments in this field depends on two main aspects:

\begin{enumerate}[label=\arabic*)]
    \item The early diagnosis of the disease.
    \item The correct and constant evaluation of the effectiveness of the treatment.
\end{enumerate}

In article [2] we proposed the method for complex
recognition of Parkinson’s disease using machine learning, based on markers of voice analysis and changes in patient movements on known data sets. The time-frequency
function, (the wavelet function) and the Meyer kepstral
coefficient function are used. The KNN algorithm and
the algorithm of a two-layer neural network were used
for training and testing on publicly available datasets on
speech changes and motion retardation in Parkinson’s disease. A Bayesian optimizer was also used to improve the
hyperparameters of the KNN algorithm. The constructed
models achieved an accuracy of 94.7 \% and 96.2 \% on a
data set on speech changes in patients with Parkinson’s
disease and a data set on slowing down the movement of
patients, respectively. The recognition results are close to
the world level. The proposed technique is intended for
use in the IoT network of IT diagnostics of PD.

\section{\textbf{KNOWLEDGE BASE AND ONTOLOGY}}
The construction of the Intelligent Parkinson’s Disease
Diagnosis System is based on the following three main
components:
\begin{enumerate}[label=\arabic*)]
    \item Knowledge Base: Collects and stores knowledge and
data about Parkinson’s Disease;
    \item Problem Solver: Utilizes information from the
knowledge base to solve specific problems;
    \item Interaction Interface: Provides a way for doctors and
patients to interact with the system.
\end{enumerate}

A. \hspace{0.1cm} Construction of the Knowledge Base for the Intelligent Parkinson’s Disease OSTIS System \\
The ontology construction of the knowledge base of
the Smart Parkinson’s Disease Diagnostic OSTIS system
is subdivided into the following areas.
\begin{SCn}
\scnheader{Parkinson's Disease Ontology}
\scnsubset{ontology}
\scnsubset{ostis-system}
\scnsubset{Familial neurodegenerative disease}
\begin{scnrelfromset}{Decomposition*:}{
    \scnitem{Basic Classification of Parkinson’s Disease}
    \scnitem{Clinical Features of Parkinson’s Disease}
    \scnitem{Parkinson-Plus Syndromes [3]}
    \scnitem{Etiology of Parkinson’s Disease}
    \scnitem{Neuropathology of Parkinson’s Disease}
    \scnitem{Information Models of Parkinson’s Disease}
    \begin{scnindent}
     \begin{scnrelfromset}{Decomposition*:}{
        \scnitem{Research Models [4]}
        \scnitem{Predictive Analytic Models}
    }
    \end{scnrelfromset}
    \end{scnindent}
}
\end{scnrelfromset}
\newpage

\scnheader{Predictive Analytic Models}
\begin{scnrelfromset}{Decomposition*:}{
    \scnitem{Medical Imaging Data [5]}
    \scnidtf{Processing MRI, PET, SPECT, etc., imaging data for the diagnosis and research of Parkinson’s Disease.}
    \scnitem{Biomarkers [6]}
    \scnidtf{Analyzing $\alpha$-synuclein, inflammatory markers, oxidative stress markers, etc., to monitor disease progression.}
    \scnitem{Genetic Information [7]}
    \scnidtf{Analyzing Single Nucleotide Polymorphisms (SNPs) and gene mutations associated with an increased risk of Parkinson’s Disease.}
    \scnitem{Clinical Feature Data Processor}
    \begin{scnrelfromset}{Decomposition*:}{
        \scnitem{Motion Data Processor}
        \scnitem{Voice Data Processor}
    }
    \end{scnrelfromset}
}
\end{scnrelfromset}
\scnheader{Voice Data Processor}
\scnidtf{Changes in speech and language in patients with
Parkinson’s Disease, such as softer voice, slower
speech, and reduced intonation variability, can be
quantified through voice analysis technology, serving
as a tool for assessing the disease and its progression.}
\begin{scnrelfromset}{Decomposition*:}{
    \scnitem{Specific Voice Data Collection Methods}
    \begin{scnrelfromset}{Decomposition*:}{
        \scnitem{Model Selector}
        \begin{scnrelfromset}{Decomposition*:}{
            \scnitem{LSTM Model}
            \scnitem{GRU Model}
            \scnitem{KNN Model}
            \scnitem{Random Forest Model}
        }
        \end{scnrelfromset}
        \scnitem{Voice Data Feature Analyzer}
        \begin{scnrelfromset}{Decomposition*:}{
            \scnitem{Voice Feature Analyzer:}
            \scnitem{Raw Voice Data Processor}
        }
        \end{scnrelfromset}
    }
    \end{scnrelfromset}
}
\end{scnrelfromset}
\scnheader{Motion Data Processor}
\scnidtf{Involves patient’s motor abilities and control, including tremors, muscle rigidity, bradykinesia, gait, and
balance issues, analyzed through motion analysis technologies like wearable devices or motion capture
systems.}
\begin{scnrelfromset}{Decomposition*:}{
    \scnitem{Specific Motion Data Collection Methods}
    \begin{scnrelfromset}{Decomposition*:}{
        \scnitem{Model Selector}
        \begin{scnrelfromset}{Decomposition*:}{
            \scnitem{LSTM Model}
            \scnitem{GRU Model}
            \scnitem{KNN Model}
            \scnitem{Random Forest Model}
        }
        \end{scnrelfromset}
        \scnitem{Motion Data Feature Analyzer}
        \begin{scnrelfromset}{Decomposition*:}{
            \scnitem{db6 Wavelet Feature Analyzer:}
            \scnitem{Raw Motion Data Processor}
        }
        \end{scnrelfromset}
    }
    \end{scnrelfromset}
}    
\end{scnrelfromset}
\end{SCn}
B.  \textit{Problem Solver for the Intelligent Parkinson’s Disease Diagnosis System} \\
The task resolver of the Intelligent Parkinson’s Disease Diagnosis System is a collective of interacting scagents that facilitate the resolution of diagnostic and management issues related to Parkinson’s disease. Herein is a fundamental decomposition of the task resolver for the Intelligent Parkinson’s Disease Diagnosis System, based on the principal sc-agent classes designated for Parkinson’s disease diagnostic purposes. 
\begin{SCn}
\scnheader{Intelligent Parkinson’s Disease Diagnosis System Problem
Solver}
\begin{scnrelfromset}{Decomposition}{
    \scnitem{Clinical Data Analysis abstract sc-agent}
    \begin{scnindent}
        \begin{scnrelfromset}{Decomposition}{
        \scnitem{Motion Data Analysis abstract sc-agent}
        \scnitem{Voice Data Analysis abstract sc-agent}
        \scnitem{Genetic Information Analysis abstract sc-agent}
        }
        \end{scnrelfromset}
    \end{scnindent}
    \scnitem{Symptom Severity Assessment abstract sc-agent}
    \scnitem{Medical Database and Diagnostic Tool Interface abstract sc-agent}
    \begin{scnindent}
        \begin{scnrelfromset}{Decomposition}{
            \scnitem{Medical Imaging Database Access abstract sc-agent}
            \scnitem{Biomarker Analysis abstract sc-agent}
            \scnitem{Time Series Analysis abstract sc-agent}
        }
        \end{scnrelfromset}
    \end{scnindent}
    \scnitem{Symptom and Treatment Plan Correlation abstract sc-agent}
    \scnitem{Diagnosis and Treatment Knowledge Base Verification abstract sc-agent}
    \begin{scnindent}
        \begin{scnrelfromset}{Decomposition}{
        \scnitem{Treatment Plan Efficacy Verification abstract sc-agent}
        \scnitem{Disease Diagnosis Accuracy Verification abstract sc-agent}
        \scnitem{Patient Data Privacy and Security Protection abstract sc-agent}
        }
        \end{scnrelfromset}
    \end{scnindent}
    \scnitem{Intelligent Question-Answering abstract sc-agent}
    \begin{scnindent}
         \begin{scnrelfromset}{Decomposition}{
             \scnitem{User Interaction abstract sc-agent}
             \scnitem{Medical Knowledge Graph}
             \scnitem{Answer Generation abstract sc-agent}
        }
        \end{scnrelfromset}
    \end{scnindent}
   
}    
\end{scnrelfromset}
\newpage
\begin{figure}
    \includegraphics[width=\linewidth]{graphics/1.png}
    \caption{Framework Diagram for the Intelligent Parkinson’s Disease OSTIS System’s Automated Diagnostic and Question-Answering Tool.}
\end{figure}
\section{\textbf{SYSTEM ARCHITECTURE OVERVIEW}}
\setlength{\parindent}{0.4cm}
This section offers a comprehensive overview of the
architectural underpinnings governing the deployment of
our Parkinson’s disease diagnosis model [2] within the
IoT framework. It expounds upon the intricate interplay
between system components designed to facilitate efficient data processing, storage, and presentation. Fig
1 illustrates an IoT system architecture enhanced with
neural network capabilities using OSTIS technology. \\
IoT Device (Client): Data collection and preprocessing
occur on the client-side. This could be a smart device
such as a sensor or a smartphone, responsible for data
acquisition and preliminary data preprocessing and feature extraction. \\
Local Flask Server [8]: The Flask server is located
locally and serves as the receiver for data from the IoT
device. It acts as an intermediary for data transmission,
forwarding the received feature data to the OSTIS server. \\
OSTIS [9] Server: The OSTIS server is a knowledge
graph platform that receives and processes data from the
local Flask server. Running on this server is a Neural
Network Predictor Agent responsible for loading internal
neural network model files. \\
Neural Network Predictor Agent: This agent is responsible for loading and executing neural network models,
processing incoming feature data, and making predictions. Predictions can be associated with knowledge
within the OSTIS system and ultimately saved to a local
database. \\
Local Database: The local database is used to store the
prediction results returned from the OSTIS server, along
with other relevant information. \\
The workflow of the entire system is as follows: \\
\begin{enumerate}[label=\arabic*)]
    \item IoT device collects and preprocesses data.
    \item Preprocessed data is sent to the local Flask server.
    \item The local Flask server forwards the data to the OSTIS server.
    \item The Neural Network Predictor Agent on the OSTIS server loads the neural network model and processes the data.
    \item Determining the subclass, type, and subtype based on the Parkinson’s disease diagnostic classifier, i. e., the types of diagnostic entities in medical ontology;
    \item Establishing the inherent attributes and characteristics of the diagnostic category;
    \item Determining the values of features for that diagnostic category;
    \item Resolving polysemy in the diagnostic process;
    \item Establishing the corresponding connections between diagnostic entities and concepts with medical semantic features in the knowledge base;
    \item Establishing relationships between diagnostic entities belonging to a specific category of Parkinson’s disease.
    \item The processed results are associated with the knowledge graph and finally saved to the local database. 
\end{enumerate} \\
This system allows real-time data processing and complex object recognition in an IoT environment, combining data with a knowledge graph to support advanced
analysis and decision-making. Proper configuration and
management of each component are required to ensure
the efficient operation of the system.
\section{\textbf{IOT DEVICE AND DATA COLLECTION}}
Within this subsection, we embark on an in-depth
exploration of the IoT device deployed for data collection.
We elucidate its pivotal role in the acquisition of two
critical data modalities: movement data and audio data.
We delve into the intricacies of data preprocessing and
transmission to the local server, underscoring the cardinal
importance of real-time data acquisition capabilities. As
shown in Figure 2, the IoT system architecture based on
neural network with OSTIS technology is illustrated. \\
First, let us introduce the IoT device used. The device
plays a key role in data acquisition by capturing two
important data types: motion data and audio data. Motion
data records the movement characteristics of Parkinson’s
patients, while audio data captures their sound characteristics. \\
To capture the acceleration data of the cell phone: \\
\begin{enumerate}[label=\arabic*)]
    \item use a third-party library in Python (PySensors) to access the phone’s acceleration sensor.
    \item set up the sensor parameters, setting the sampling rate to 64hz and the sensor precision to 16 bits.
    \item create a data storage structure to hold the acquired acceleration data.
    \item In a loop, periodically read the acceleration sensor data and store it in the data structure.
    \item Store the data in a local file for further processing and analysis.
\end{enumerate}
Collect the voice data from the cell phone:
\end{SCn}
\end{document}