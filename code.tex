\documentclass{article}
\usepackage[english]{babel}
\usepackage{multicol}
\usepackage{import}
\usepackage{scn}
\usepackage[
	a4paper,
	top= 1.5 cm,
	bottom=1.5cm,
	left=2.5cm,
	right=2.5cm,
]{geometry}

\usepackage{graphicx}
\usepackage{float}
\newcommand{\customsize}{\fontsize{7}{7}\selectfont}

\setcounter{page}{159}

\begin{document}
\begin{multicols}{2}
\begin{enumerate}
\item[2)]Medesk — this cloud-based medical information
system is used in more than 20 countries around
the world and in 72 regions of Russia. It is suitable
for healthcare institutions of any size, from private
clinics to networks. electronic medical records,
CRM, telemedicine and solutions for managers and
physicians are available in Medesk.
\item[3)] KMIS is a complex MIS suitable for automation
of any medical institutions, the feature of which is
integration with Federal systems.
\item[4)]MEDMIS is a relatively young MIS that entered
the market in 2017. In 4 years, MEDMIS has
been used by more than 200 medical organizations.
MEDMIS is constantly evolving and gaining momentum: updates are released once a week.
 \item[5)]MedAngel is an MIS with the possibility of individual customization for the specifics of the clinic’s
work. There is only a boxed version of the program
with open code. The system is modular, you can
assemble a personalized kit.
\end{enumerate}
The list represents the most popular systems on the
market. Each system has its own advantages and features,
and the choice depends on the needs of a particular
healthcare facility.

An Intelligent Medical System (IMS) is an information
system that uses artificial intelligence techniques and
approaches, including semantic technologies, to process,
store and analyze medical data. Such systems represent
a key element in modern healthcare, providing effective
management of patient information, improved decisionmaking in medical practice and personalized patient care.

With the rapid development of artificial intelligence
systems, IMS are becoming increasingly in demand.
Their ability to analyze and interpret large amounts of
data allows to reduce the workload of medical personnel, improve diagnostic accuracy and optimize treatment
processes.

However, problems arise due to the variety of formats
for storing medical data in different countries. Norms and
legislation governing the processing and protection of
medical information may differ from country to country.
For example, the Republic of Belarus, Russia and the
Republic of Kazakhstan have different standards and
requirements for storing and processing medical data.
This can create difficulties in integrating and sharing
information between systems, as well as increase the
risk of breaches of patient data privacy and security. For
successful implementation of intelligent medical systems,
it is necessary to take these differences into account and
develop appropriate mechanisms for data standardization
and interoperability.
\begin{center}
    III. Proposed approach
\end{center}

At the heart of any knowledge-based IMS is a formalized knowledge base, the quality and volume of
which directly affects the effectiveness of the system. For
uniform representation and unambiguous interpretation
of knowledge bases, a common set of all terms used
in practice is required. Terminology should be generally
accepted and understandable to medical specialists, i.e.
it should be the result of ontological agreement in the
field of medicine.

Ontological modeling of the subject domain includes
specification, conceptualization and formalization. At the
specification stage, a glossary of terms is built, including
all terms important for the subject area and their descriptions. In the conceptualization stage, important objects of
the subject domain are identified. In this stage, the hierarchy and relationships between the objects of the subject
domain are also defined. In the formalization stage,
meta-objects and relationships between meta-objects are
created that correspond to the objects and relationships
between the objects of the subject domain. As a result,
an ontology of the subject domain will be obtained.
At the actualization stage, object parameters and their
domains are defined (parameter domain is the area of
acceptable parameter values), as well as values, classes,
subclasses and class instances. Parameters, parameter
domains, parameter values, classes, subclasses and class
instances are realized as metaobjects (a metaobject is
some text representing a definition, concept or some
other description.) of appropriate types. After the work
at the actualization stage, the ontology is turned into a
knowledge base. The result of ontology modeling is the
ontology of the subject area. To turn the ontology into a
knowledge base, it is necessary to actualize it.
\par In MIS, the problem of incompatibility of data formats
is a significant challenge that can hinder effective information sharing and interoperability between different
healthcare systems. OSTIS (Open Semantic Technology
for Intelligent Systems) is proposed to address this problem. OSTIS provides innovative tools and approaches for
creating semantically interoperable medical systems that
original format and structure.
\par One of the key features of OSTIS technology is its
ability to unify the representation of different types
of knowledge in a single database. This enables the
creation of centralized information repositories where
medical data can be organized and structured according
to common semantic standards, while ensuring a high
degree of compatibility and interoperability.
\par OSTIS technology is also highly flexible and modifiable, allowing the system to be customized to meet
the specific requirements and standards of each country,
including the Republic of Belarus, the Russian Federation
and the Republic of Kazakhstan. As a result, medical information systems based on OSTIS technology can easily
adapt to various regulatory and legislative requirements,
which ensures their seamless integration into the existing healthcare infrastructure.
\par One of the key benefits of using OSTIS technology is the ability to automatically convert and compare data from different formats. This helps eliminate interoperability issues and ensures seamless information exchange between different medical systems and institutions, which ultimately improves system efficiency and the quality of care provided. 
\par Thus, the use of OSTIS technology is an effective and promising approach to solving the problem of incompatibility of data formats in medical information systems. It ensures the creation of modern and innovative healthcare systems capable of adapting to a variety of requirements and changes in the medical field, which is a key factor for improving the quality and accessibility of medical care in different countries. 
\par As an illustrative example, let us give the possibilities of using OSTIS formalization technology for two subject areas related to intelligent medical systems.
\par Formalizing the subject matter of a patient’s medical record using OSTIS technology demonstrates significant convenience and efficiency in processing and storing medical data. The formalization process allows describing various aspects of a patient’s health, including medical history, diagnoses, treatments and symptoms of diseases. \par One of the features of formalization on OSTIS technology is the ability to create a single semantic model that integrates different aspects of medical information in a unified format. For example, information about gastrointestinal (GI) diseases can be organized in the form of semantic entities such as disease type, symptoms, diagnosis and treatment methods. 
\par The advantage of using OSTIS technology is that information on GI diseases can be stored in a single semantic format, making it easier to access and share data between different medical systems and institutions. In addition, thanks to the semantic approach, information about diseases and their symptoms can be structured and categorized for easy analysis and processing. 
\par Formalizing the subject matter of a patient’s medical record using OSTIS technology demonstrates significant convenience and efficiency in processing and storing medical data. The formalization process allows describing various aspects of a patient’s health, including medical history, diagnoses, treatments and symptoms of diseases. \par One of the features of formalization on OSTIS technology is the ability to create a single semantic model that integrates different aspects of medical information in a unified format. For example, information about gastrointestinal (GI) diseases can be organized in the form of semantic entities such as disease type, symptoms, diagnosis and treatment methods.
\par  The advantage of using OSTIS technology is that information on GI diseases can be stored in a single semantic format, making it easier to access and share data between different medical systems and institutions. In addition, thanks to the semantic approach, information about diseases and their symptoms can be structured and categorized for easy analysis and processing. 
\begin{center}
  IV. The subject area and ontology of a patient’s medical record   
\end{center}
\par Medical records are kept according to the order approved by the Ministry of Health. In the Republic of Belarus, this is Order No. 710 of the Ministry of Health of the Republic of Belarus of August 30, 2007 "On Approval of the Forms of Primary Medical Documentation in Outpatient and Polyclinic Organizations". The card of a patient receiving medical care in outpatient conditions must comply with the form 025/-07 "Medical card of an outpatient". In the Russian Federation it is the Order of the Ministry of Health of the Russian Federation from December 15, 20144 № 834n "On approval of the unified forms of medical documentation used in medical organizations providing medical care in outpatient settings, and procedures for filling out the form № 025/u "Medical card of a patient receiving medical care in outpatient settings". In the Republic of Kazakhstan it is the Order of the Acting Minister of Health of the Republic of Kazakhstan from October 30, 2020 № KR DSM- 175/200 "On approval of forms of record documentation in the field of health care" form № 052/u "Medical card of an outpatient". There are similarities and differences. There is a common part: full name, sex, date of birth, place of work, blood group and Rh. There are also differences, for example, in the RF there is a section SNILS, in the RK IIN, and in the RB personal number on the passport [9]– [11]. 
\par Based on OSTIS technology, it is possible to effectively formalize various concepts from a patient’s medical record, taking into account the peculiarities of regulatory documentation in different countries. Despite the differences in the formats and structure of medical documents in the Republic of Belarus, the Russian Federation and the Republic of Kazakhstan, OSTIS technology can be used to create a universal model for data storage and processing, ensuring their standardization and flexibility.
\par The advantage of OSTIS technology lies in its graphbased approach to information representation, which allows for quick changes to the system without changing its structure, just modifying the knowledge base. This approach makes the system flexible and adaptable to new requirements and changes in legislation, which is especially important in the medical field, where requirements and standards are constantly changing. 
\par Fig. 1 shows a fragment of a patient medical record ontology formalized using SCG language and including such basic concepts as: 
\end{multicols}
\newpage
\begin{figure}[H]
    \centering
\includegraphics[width=1\linewidth]{Снимок экрана (9).png}
    \caption{Fragment of medical record ontology}
\end{figure}
\begin{multicols}{2}

 For example, patient information, including personal data and medical history, can be represented in graph structures where each node represents a different concept and the links between nodes reflect the relationships between the data. This approach makes it easy to add new data or modify existing data without having to redesign the entire system. 
\par Using OSTIS technology, it is also possible to bring different formats and structures of data from medical documents to a common standard of presentation. For example, patient information from different countries may contain different fields and formats, such as SNILS in Russia, IIN in Kazakhstan, and personal passport number in Belarus. However, thanks to the flexibility and adaptability of OSTIS technology, these data can be standardized and combined into a common information model, providing a unified view and access to a patient’s medical information regardless of where they live or where they are being treated.
\par In addition, with the help of OSTIS technology, metainformation can be added to the patient’s medical data at the level of knowledge base, which allows its use by intelligent system agents in the process of processing and analysis. This makes it possible to create a more flexible and intelligent system of medical data processing than just adding comments in a classical medical information system. An example of a fragment of such data formalization is presented in Fig. 2.  
\begin{figure}[H]
    \centering
\includegraphics[width=1\linewidth]{Снимок экрана (10).png}
    \caption{A fragment of the ontology of a medical record that allows you to store clarifying information }
\end{figure}
\par Thus, the use of OSTIS technology to formalize data from patients’ medical records and information on GI diseases allows to create a flexible and adaptive system that can effectively take into account new requirements and changes in legislation, as well as standardize the presentation of information to simplify its analysis and processing.  
\end{multicols}
\end document
